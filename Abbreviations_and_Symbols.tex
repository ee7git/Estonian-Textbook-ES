\section{Abreviaturas y Símbolos}

\bigskip
\begin{tabular}{ l l }
	\begin{tabular}{ l c l }
	abbr.	& = & abreviatura \\
	abess.	& = & abesivo \\
	abl.	& = & ablativo \\
	adess.	& = & adesivo \\
	adj.	& = & adjetivo \\
	adv.	& = & adverbio \\
	all.	& = & alativo \\
	comit.	& = & comitativo \\
	comp.	& = & comparativo \\
	conj.	& = & conjunción \\
	cont.	& = & continuado \\
	dim.	& = & diminutivo \\
	e.g.	& = & por ejemplo \\
	elat.	& = & elativo \\
	emph.	& = & enfático \\
	etc.	& = & etcétera \\
	gen.	& = & genitivo \\
	i.e.	& = & en esencia \\
	ill.	& = & ilativo \\
	imper.	& = & imperativo \\
	imperf.	& = & imperfecto \\
	indecl.	& = & indeclinable \\
	iness.	& = & inesivo \\
	inf.	& = & infinitivo \\
	interj.	& = & interjección
	\end{tabular}
&
	\begin{tabular}{ l c l }
	lit.	& = & literalmente \\
	n.		& = & sustantivo \\
	neg.	& = & negativo \\
	nom.	& = & nominativo \\
	num.	& = & número \\
	part.	& = & partitivo \\
	partic.	& = & participio \\
	pass.	& = & pasivo \\
	perf.	& = & perfecto \\
	pers.	& = & persona \\
	pl.		& = & plural \\
	postp.	& = & posposición \\
	prep.	& = & preposición \\
	pres.	& = & presente \\
	pron.	& = & pronombre \\
	refl.	& = & reflexivo \\
	sing.	& = & singular \\
	superl.	& = & superlativo \\
	term.	& = & terminativo \\
	transl.	& = & translativo \\
	v.		& = & verbo \\
	v.i.	& = & verbo intransitivo (no toma ningún objeto) \\
	vs.		& = & versus \\
	v.t.	& = & verbo transitivo (toma un objeto)
	\end{tabular}
\end{tabular}\\[1cm]

\textasciiacute \qquad indica que la tensión está en una sílaba dada, en contraste con el patrón habitual de hacer hincapié en la primera sílaba.\\

\textasciigrave \qquad indica un sonido extra largo (tercer grado) en la sílaba que sigue.\\

\textquotesingle \qquad indica palatalización de consonante.\\

\S \qquad significa sección.