\section{Prólogo}

Durante siglos, los estonios han tenido contacto cercano con otras nacionalidades que viven en la zona del Mar Báltico al noreste de Europa. Entre las dos Guerras Mundiales (cuando Estonia era independiente), se fortalecieron los contactos políticos, económicos y culturales con los países vecinos. Incluso los contactos personales se desarrollaron, en gran medida, debido al turismo. Los mismos procesos se haces aún más evidentes hoy en día, después de que Estonia recuperara su independencia de la Unión Soviética en 1991.\\

A finales de la Segunda Guerra Mundial, decenas de miles de estonios huyeron a Suecia y Alemania. Muchos de ellos se instalaron en los Estados Unidos de América, Canadá y Australia. Fueron recibidos con amistad y entendimiento en los países donde buscaron refugio. En sus actividades laborales y de ocio diarios, los estonios se adaptaron bien a la vida en otros países, y la mayoría de ellos son ahora ciudadanos de sus nuevas patrias.\\

Los estonios en el extranjero no han olvidado su origen o su lengua. Quieren preservar su patrimonio cultural y mantener sus tradiciones. La colección de folclore de Estonia, por ejemplo, es uno de los más grandes del mundo, y los estonios en el extranjero con orgullo cuentan a sus hijos y amigos sobre el legendario héroe cuyas hazañas se registran en el épico folclore Kalevipoeg. Hay incluso una extensa y muy rica variedad de literatura estonia moderna, aunque poco de ella ha sido traducida a otros idiomas aún. Músicos y cantantes de Estonia son excepcionales, y grandes festivales en su patria así como en el extranjero reúnen a miles de miembros del coro, bailarines folclóricos, y gimnastas rítmicos para ganar fama por sus actuaciones. Anticuadas artesanías estonias también han llamado la atención - particularmente las artes textiles, cuero, madera y orfebrería. Muchos artistas y estudiosos contemporáneos han ganado reconocimiento internacional, por trabajos relacionados o inspirados por las viejas tradiciones y nuevos desarrollos en Estonia.\\

La lengua estonia pertenece a la familia ugrofinesa, junto con el finés, el húngaro, las lenguas sami (lapón), y un número de otras lenguas habladas por los pueblos dispersos en el norte de Rusia. Las lenguas de los pueblos cercanos - ruso, letón, lituano, sueco - se encuentran en un grupo diferente, llamada la familia indoeuropea. El inglés también se encuentra en esta última familia, lo que significa que su estructura difiere en importantes aspectos de la lengua estonia. Sin embargo, no es tan difícil para una persona de habla inglesa aprender estonio como generalmente se cree.\\

Este libro está destinado principalmente para los estadounidenses y otros hablantes del inglés que, por diversas razones, están interesados en el idioma estonio. En la preparación de este libro, sin embargo, el autor también tuvo en cuenta la generación más joven de los estonios residentes en el extranjero, sin la oportunidad de aprender la gramática del estonio en las escuelas a las que asisten.\\

El libro se puede utilizar para el estudio independiente, pero para un aprendizaje óptimo, se recomienda la ayuda de una persona de habla estonia, sobre todo al principio. En caso de que el libro se utilice en un curso, el instructor puede cambiar el orden de los temas o añadir más ejercicios según sea necesario.\\

El libro contiene 40 lecciones, cada una de las cuales tiene seis secciones: gramática, texto (selección de lectura), vocabulario, ejercicios (diseñado para reforzar el aprendizaje tanto de gramática como de vocabulario), expresiones (elegidas acorde a la gramática del capítulo en mente, y a menudo agrupados por tema), y las respuestas a los ejercicios.\\

En las secciones de gramática, el autor ha tratado de presentar las principales características de la gramática estonia de la forma más sencilla posible. Para hacer las cosas más claras, las comparaciones con las reglas de la gramática inglesa se hacen a menudo.\\

Aquellos que no tienen deseo ni tiempo para un estudio profundo de la gramática estonia pero que desean ampliar su repertorio de frases coloquiales y palabras comunes encontrará el tema de las expresiones de cada lección (Saludos y Agradecimientos, Comidas y Bebidas, Tiempo, Clima, Correspondencias, Enfermedad , Oficio, etc.) listadas en la tabla de contenidos. El índice también enumera los temas tratados en las lecturas y las listas de expresiones.\\

Las selecciones de lectura, mayormente composiciones originales hechas por el autor, están diseñadas para reforzar los puntos de la gramática. Las cursivas se utilizan para identificar las palabras o frases que ilustran las formas gramaticales presentadas en la misma lección. Al mismo tiempo, el autor ha tratado de cubrir un amplio rango de temas y situaciones para construir el vocabulario del alumno tanto como sea posible para una conversación regular.\\

Al final del libro se incluye un diccionario general de Estonia-Inglés, con todos los términos presentados en la lista de vocabulario de cada capítulo y otras palabras comunes. Para la traducción de palabras del inglés al estonio, el estudiante necesitará un diccionario Inglés-Estonio. Una breve revisión de los términos gramaticales también se presenta al final. El índice consta de dos partes, con listados alfabéticos separados en términos gramaticales y temas de conversación.\\

La idea de escribir un libro sobre el estonio se presentó en el verano de 1960 en el curso de la Escuela de Continuación Estonia (Estniska Folkhögskolan) en Gimo, Suecia, donde el autor enseñó estonio por varios años y por lo mismo obtuvo la perspicacia sobre las dificultades de instruir a los jóvenes sin previo estudio de la gramática estonia.\\

La fuente de inspiración fue ombudsman Nils Hellstrom, y estoy muy agradecido con él, no sólo por surgir con la idea de este libro, sino también por organizar su primera publicación (en 1962) a través de Bokförlaget Medborgarskolan. También deseo expresar mis agradecimientos al director Henry Jarild, por su cooperación y ayuda en relación con la publicación original de este libro.\\

Quiero expresar un especial agradecimiento a Gita Aasmaa y Tarmo Oja, no sólo por proporcionar diversas formas de asistencia técnica, sino también por contribuir puntos de vista muy valiosos y sugerencias con respecto al contenido del libro.\\

Estoy especialmente agradecido con el profesor Ain Haas por asumir y llevar a cabo extremadamente bien la enorme tarea de traducir y actualizar el libro de esta edición.\\

\hfill \textit{Juhan Tuldava}

\hfill Tartu, Estonia
