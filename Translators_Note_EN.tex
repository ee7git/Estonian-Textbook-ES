\section*{Nota del Traductor}

\subsection*{Versión Inglés}

A number of books are now available for people wishing to study the Estonian language. Anyone serious about this should not overlook Prof. Tuldava's work. In the effort to improve my own command of Estonian, I found nothing more valuable than his book. It is truly impressive in its clarity, thoroughness, and logic of progression. It provides a rigorous course of instruction, equivalent to two years of collage work, but has many interesting twists and even humorous touches that make the lessons enjoyable.\\

After discovering the book during a research trip to Sweden, I came to feel that it deserved a much wider distribution. As an Estonia born in Sweden and raised on the United States, with fluency in all three languages, I found myself in a good position to prepare an English version of this excellent book, for the benefit of relatives and friends in the U.S. who were unable to make use of the Swedish version. Because of their encouragement, as well as a growing number of requests from other parties, I have decided to make my translation available to a wide audience. The rising interest in the language among all sort of people with no ancestral connection to the country reflects Estonia's new status as a trendsetting Est European country making rapid strides in overcoming the legacy of Soviet occupation.\\

The English version is basically the same as the Swedish version, but some changes were made in preparing this translation. I substituted references to America names, locations, currency, etc. for many of the original Swedish ones, and took the liberty of adding some expressions I had found in my examination of Estonian dictionaries and literature. Points of grammar applicable to the Swedish language but not English have been removed, and new comments about English grammar have been added. To make it easier to use the book for independent study, I developed a more complete set of answers to the exercises and expanded the glossary.\\

During a stint as visiting professor of sociology at Tartu University in the spring of 1993, I took advantage of the opportunity to meet with Prof. Tuldava, who recently retired as head of the Germanic Language Department there. We discussed how to update and improve on the original Swedish publication. His suggestions--including a few new points of grammar that reflect changing usage in Estonia--have been incorporated in the version.\\

I wish to express special thanks to Prof. Tuldava and to my mother, Elly (Ratas) Haas, for checking the manuscript carefully at various stages and offering many helpful suggestions.\\

It is with great pleasure and pride that I offer this translation of Prof. Tuldava's book, to all those looking for a key to unlock the mysteries of the Estonian language and open the door to a hidden world of intriguing folklore, fine literature, and enjoyable conversations. There are over a million speakers of Estonian in the world today, and it is my fond hope that some more will be encouraged to join their ranks as result of this book.\\

\textit{Ain Haas}\\
Indianapolis, USA.