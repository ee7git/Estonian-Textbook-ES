% Lesson 7

\chapter{Séptima Lección} % Chapter title

\label{ch:lesson07} % For referencing the chapter elsewhere, use \autoref{ch:examples} 

%----------------------------------------------------------------------------------------


% =====================
% 		GRAMATICA
% =====================
\Large{\section*{Gramática}}

\S\ 31. Todos los sustantivos, adjetivos, pronombres y números en estonio tienen 14 casos diferentes, tanto en singular como en plural. Sin embargo los principiantes no deben ser disuadidos por este hecho. En realidad, hay sólo unos pocos casos básicos en que se basa el resto del sistema de casos. Como verán, el sistema de casos es muy regular. Las terminaciones son constantes y se añaden a la raíz de la palabra sin cambiar ésta, por lo que es fácil distinguir entre la raíz y la terminación. 

\Large{\subsection*{Genitivo Singular}}

\S\ 32. Una de los casos básicas es el genitivo singular, que se utiliza sobre todo para indicar el \emph{poseedor} o propietario de algo. En español, esto se indica mediante la utilización de la preposición `de': de el hombre, de el niño, etc. \\

\begin{center}
\begin{tabular}{ l l }
	\bemph{Kelle} raamat? \bemph{Lapse} raamat.	& `¿El libro de quién? El libro del niño.' \\
	\bemph{Mille} kaas? \bemph{Raamatu} kaas.	& `¿La cubierta de qué? La cubierta del libro'
\end{tabular}
\end{center}
\bigskip

\S\ 33. El estonio utiliza el genitivo también para representar origen o asociación: \bemph{Linna} tänavad `las calles de la ciudad', \bemph{maja} katus' `el techo de la casa', \bemph{olukorra} peremees `el maestro de la situación'. \\

En estonio, dos genitivos pueden estar lado a lado: \bemph{poisi venna} raamat `el libro del hermano del niño'. \\

\S\ 34. Tenga en cuenta que un adjetivo que modifica a un sustantivo en genitivo singular coincide con el sustantivo. En otras palabras, el adjetivo debe estar también en el caso genitivo singular: \bemph{väikese} [\emph{gen.} cantar.] \bemph{lapse} [\emph{gen.} niño.] raamat `El libro de el pequeño niño'. \\

\S\ 35. El genitivo singular siempre termina en vocal. No hay reglas establecidas para saber en qué vocal una palabra termina en el caso genitivo singular. En nuestras listas de palabras, el genitivo se dará después de la forma nominativa de la palabra. Por ejemplo:

\begin{center}
\begin{tabular}{ l l }
	\bemph{vend, venna}				& hermano, del hermano \\
	\bemph{raamat, -u [raamatu]}	& libro, del libro \\
	\bemph{nai/ne, -se [naise]}		& mujer, de la mujer \\
	\bemph{maja, - [maja]}			& casa, de la casa
\end{tabular}
\end{center}
\bigskip

\S\ 36. Usted tendrá que aprender estas dos formas cuando se encuentra con una nueva palabra. Si conoce la forma del genitivo, puede construir todas las restantes formas singulares de casos (excepto el partitivo), e incluso el nominativo plural, simplemente agregando una terminación a la forma del genitivo. Algunos ejemplos:

\begin{center}
\begin{tabular}{ l l l }
	\emph{Nominativo singular} 	& \emph{Genitivo singular} 	& \emph{Otros casos} \\
	\hline
								&							& \\ 
	raamat `libro'				& raamatu `del libro'		& raamatu/\bemph{s} `en el libro' \\
								& 							& raamatu/\bemph{ga} `con el libro' \\
								& 							& raamatu/\bemph{ta} `sin el libro' \\
								& 							& raamatu/\bemph{d} `los libros' \\
								&							& \\
	müts `gorro'				& mütsi `del gorro'			& mütsi/\bemph{s} `en el gorro' \\
								& 							& mütsi/\bemph{ga} `con el gorro' \\
								& 							& mütsi/\bemph{ta} `sin el gorro' \\
								& 							& mütsi/\bemph{d} `gorros'
\end{tabular}
\end{center}
\bigskip

\S\ 37. Las terminaciones -s, -ga, -ta, que se añaden a la forma del genitivo, tienen el mismo significado que las preposiciones `en, con y sin' en español. En el estonio, sin embargo, estas terminaciones nunca pueden estar separadas como palabras separadas. \\

\S\ 38. Una palabra que termine con una consonante en el nominativo singular siempre termina con uno de los cuatro vocales (a, e, i, u) en el genitivo singular: \\

\begin{center}
\begin{tabular}{ l l }
	\emph{Nominativo singular} 	& \emph{enitivo singular} \\
	linn `ciudad'				& linn/\bemph{a} `de la ciudad' \\
	ilm `tiempo, clima' 		& ilm/\bemph{a} \\
	ilus `lindo/a'				& ilus/\bemph{a} \\
	laps `niño/a'				& laps/\bemph{e} \\
	noor `joven'				& noor/\bemph{e} \\
	suur `grande'				& suur/\bemph{e} \\
	uks `puerta'				& uks/\bemph{e} \\
	hääl `voz' 					& hääl/\bemph{e} \\
	kool `escuela'				& kool/\bemph{i} \\
	pliiats	`lápiz'				& pliiats/\bemph{i} \\
	tool `silla'				& tool/\bemph{i} \\
	tüdruk `niña'				& tüdruk/\bemph{u} \\
	laul `canción'				& laul/\bemph{u} \\
	raamat `libro'				& raamat/\bemph{u} \\
	õpik `manual'				& õpik/\bemph{u} \\
	suits `humo'				& suits/\bemph{u}
\end{tabular}
\end{center}
\bigskip

\bemph{lapse} raamat `el libro del/de la niño/a', \bemph{tüdruku} pliiats `el lápiz de la niña'

\S\ 39. Tenga en cuenta que los nombres extranjeros que terminan en una consonante generalmente toman la terminación \bemph{-i} en el genitivo singular: New York, New Yorgi, Washington, Washingtoni, Londres, Londoni; Johnson, johnsoni; Smith, Smithi. Ejemplos: \\

\begin{center}
	\bemph{Bostoni} sadam `el puerto de Boston'
	\bemph{Hoffmani} korter `el apartamento de Hoffman’
\end{center}
\bigskip

Los nombres extranjeros que terminan en \bemph{s} a veces toman la terminación \bemph{-e} en el genitivo singular: Indianapolis, Indianapolise, Los Angeles, Los Angelese, Buenos Aires, Buenos Airese; Celsius, Celsiuse. \\

\S\ 40. Una palabra que termina en vocal en el nominativo singular por lo general mantiene la misma vocal para la terminación en el genitivo singular, es decir, la palabra sigue siendo la misma: \\

\begin{center}
\begin{tabular}{ l l }
	 \bemph{Isa} loeb [\emph{nom. sing.}] 	& `El padre lee' \\
	 \bemph{Isa} raamat [\emph{gen. sing.}]	& `El libro del padre' 
\end{tabular}
\end{center}
\bigskip

Estos son algunos otros ejemplos de palabras en que los casos nominativo y genitivo singular son el mismo: \\

\bemph{ema} `madre', \bemph{tädi} `tía', \bemph{onu} `tío', \bemph{proua} `Sra.', \bemph{härra} `Sr.', \bemph{preili} `Señorita', \bemph{õpetaja} `profesor', \bemph{töö} `trabajo', \bemph{vana} `viejo/a', \bemph{hea} `bueno, bien' \\

\bemph{esti} `Estonia', \bemph{Ameerika} `America', \bemph{Rootsi} `Sweden', \bemph{Helsingi} `Helsinki',
\bemph{Oslo}, \bemph{Tartu}, \bemph{Narva}. \\

Hay, sin embargo, un número pequeño de excepciones, como \bemph{nimi} `nombre' -- \emph{gen.} \bemph{nime}; \bemph{meri} `mar' -- \emph{gen.} \bemph{mere}; \bemph{veri} `sangre' -- \emph{gen.} \bemph{vere}. En algunos casos, también puede haber un cambio de sonido en la raíz (ver \autoref{ch:lesson08}). \\

\S\ 41. La mayoría de las palabras que terminan en \bemph{-ne} en el nominativo singular toman la terminación \bemph{-se} en el genitivo singular. \\

\begin{center}
\begin{tabular}{ l l }
	inime/ne `persona' 				& inime/\bemph{se} `de la persona' \\
	milli/ne `cuál' 				& milli/\bemph{se} \\
	eestla/ne `Estonio (persona)' 	& eestla/\bemph{se} \\
	ameerikla/ne `Americano' 		& ameerikla/\bemph{se} \\
	nai/ne `mujer'					& nai/\bemph{se}
\end{tabular}
\end{center}
\bigskip

Algunas palabras de dos sílabas que terminan en \bemph{-ne} no cambian: \\

\begin{center}
\begin{tabular}{ l l }
	kõne `discurso' 	& kõne \\
	hoone `edificio' 	& hoone \\
	laine `ola' 		& laine
\end{tabular}
\end{center}
\bigskip

% ==================
% 		TEXTO
% ==================
\bigskip
\Large{\section*{Texto}}

Kelle raamat see on? See on isa raamat. Kelle maja see on? See on onu maja. Kelle korter see on? See on härra Palmi korter. Härra Palm on noor kirjanik. Noore kirjaniku uus romáan ilmub varsti. Romaani tegevus toimub maal. \\

Laps jookseb väljas. Lapse vanemad on tööl. Kool asub lähedal. Kooli hoone on uus ja ilus. Stockholmi ülikooli uus hoone saab varsti valmis. See väike poiss on Stockholmi eesti algkooli õpilane. Siin on eesti keele õpik. Kuidas sulle meeldib eesti keel? \\

Kelle auto see on? See on härra Kivisaare auto. Auto uks on laht. Astu sisse! Ära sõida nii kiiresti! Sõida aeglaselt! \\

Kuidas on õpetaja tervis? Õpetaja on vana ja haige. Haige inimese tuju on halb. Sa oled noor ja terve. Terve inimese tuju on hea.
Noore inimese elu on huvitav. Kas see harjutus on raske!? \\

Mõistatus: Isa laps ja ema laps, kuid ta pole ei ühegi inimese poeg. \\
(Tütat) 

% =======================
% 		VOCABULARIO
% =======================
\bigskip
\Large{\section*{Vocabulario}}

\begin{tabular}{ l l }
	aeglaselt			& lentamente \\
	algkool, -i			& escuela básica, escuela elemental \\
	asuma, asu/n		& estar ubicado/a \\
	auto, -				& automóvil \\
	eesti, -			& estonio [\emph{adj.}] \\
	ei ühegi			& de nadie \\
	ei ühegi inimese	& de ninguna persona \\
	haige, -			& enfermo [\emph{adj.}], persona enferma [\emph{n.}] \\
	haijutus, -e		& ejercicio \\
	hoone, -			& edificio \\
	härra				& Sr. \\
	ilmuma, ilmu/n		& aparecer, salir, publicar \\
	inime/ne, -se		& persona \\ 
	jooksma, jookse/n	& correr \\
	keel, -e			& lenguaje \\
	kelle				& de quién \\
	küresti	fast, 		& rápidamente \\
	kirjanik, -u		& escritor \\
	Kivisaar, -e		& Piedra-isla [nombre] \\
	kool, -i			& escuela \\
	korter, -i			& apartamento \\
	lähedal				& cerca (de) \\
	maal				& en el país \\
	meeldima, meeldi/n	& apelar (a) \\
	raske, -			& pesado, difícil \\
	romáan, -i			& novela [\emph{n.}] \\
	saa/n valmis		& (Yo) me preparo \\
	sulle			 	& a tí \\
	sõitma, sõida/n		& manejar, montar \\
	tegevus, -e			& actividad, operación \\
	terve, -			& sano, bien \\
	tervis, -e			& salud \\
	toimuma, toimu/n	& suceder, tomar lugar \\
	tuju, -				& humor, actitud \\
	tööl				& en el trabajo \\
	valmis				& listo \\
	varsti				& pronto \\
	väljas				& al aire libre, fuera \\
	ühe(gi)				& de uno \\
	ülikool, -i			& universidad
\end{tabular}

% ======================
% 		EJERCICIOS
% ======================
\bigskip
\Large{\section*{Ejercicios}}

\begin{enumerate}
	\item \emph{Traducir al estonio:} ¿De quién es esta casa? Es la casa del Sr. Johnson. Este es el libro de la niña. La hermana de la madre vive en el país. El hijo del profesor va a [asiste] la escuela. Este es un ejercicio difícil. La vida es interesante. La madre de la niña pequeña está en el trabajo. Ustedes son unas personas interesantes. ¿Cuándo sale el nuevo libro del autor? ¿Está el padre enfermo? No, el padre está saludable. ¡Dame el manual de la lengua estonia!

	\item \emph{Traducir estos pares de palabras contrastantes al español:} \\
	\begin{tabular}{ l l }
	noor -- vana	& haige -- terve \\
	uus -- vana		& õige -- vale \\
	suur -- väike	& kiiresti -- aeglaselt \\
	hea -- halb		& kõvasti -- tasa 
	\end{tabular}
\end{enumerate}

% ============================
% 		EXPRESIONES DE ...
% ============================
\bigskip
\Large{\section*{Expresiones de Conversación Telefónica}}

\bemph{Telefonikõne} \\ \medskip

\noindent
\bemph{Hallo, kas härra/proua/preili Kivisaar on kodus?} \\
\bemph{Ma kuulen.}  \\
\bemph{Üks silmapilk, palun.} \\
\bemph{Palun oodake. Ta tuleb kohe. Kahjuks (ta) ei ole kodus.} \\
\bemph{Kui kahju!} \\
\bemph{Vabandust, kes räägib?} \\
\bemph{Kas ta tuleb varsti tagasi?} \\
\bemph{Millal tuleb härra Kivisaar koju? Kahjuks ma ei tea.} \\
\bemph{Ta tuleb varsti.} \\
\bemph{Helistage homme uuesti.} \\
\bemph{Helista hiljem.} \\ \bigskip

Conversación telefónica \\ \medskip

\noindent
Hola, ¿está el/la Sr./Sra./Srta Kivisaar en casa? \\
Sí, ese soy yo. [\emph{lit.:} Estoy escuchando] \\
Un momento, por favor. \\
Por favor espere. Vendrá enseguida. Desafortunadamente no está en casa. \\
¡Qué lástima! [¡Qué pena!] \\
Perdón, ¿quién habla? \\
¿Va a volver pronto? \\
¿Cuándo vuelve el Sr. Kivisaar a casa? Por desgracia, no lo sé. \\
Él viene pronto. \\
Llame de nuevo mañana. \\
Llama después.

% ======================================
% 		RESPUESTA A LOS EJERCICIOS
% ======================================
\bigskip
\Large{\section*{Respuesta a los ejercicios}}

\begin{enumerate}
	\item Kelle maja see on? See on härra Johnsoni maja. Siin on tüdruku raamat. Ema õde elab maal. Õpetaja poeg käib koolis. See on raske harjutus. Elu on huvitav. Noore tüdruku ema on tööl. Teie olete huvitav inimene. Millal ilmub kirjaniku uus raamat? Kas isa on haige? Ei, isa on terve. Anna mulle eesti keele õpik!

	\item 
	\begin{tabular}{ l l }
	 joven -- viejo 	& enfermo -- saludable \\
	 nueva -- vieja 	& derecha -- malo \\
	 grande -- pequeño	& rápidamente -- lentamente \\
	 buena -- mala 		& fuerte -- despacio (sonido) 
	\end{tabular}
\end{enumerate}

%----------------------------------------------------------------------------------------