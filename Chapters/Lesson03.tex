% ======================================
%
% 				Lesson 3
%
% ======================================

\chapter{Tercera Lección} 

\label{ch:lesson03} 

%----------------------------------------------------------------------------------------

% =====================
% 		GRAMATICA
% =====================
\Large{\section*{Gramática}}

\S\ 14. La raíz del tiempo presente de un verbo, carente de terminaciones (por ejemplo, \textbf{tule-} \textless\ tule/n, \textbf{loe-} \textless\ loe/n, \textbf{räägi-} \textless\ räägi/n), puede ser utilizada de las siguientes maneras:

\begin{itemize}
	\item como la forma imperativa (ordenes) para la segunda persona singular

	\begin{center}
	\begin{tabular}{ l l }
		\textbf{tule!}	& ¡ven! \\
		\textbf{loe!}	& ¡lee! \\
		\textbf{räägi!}	& ¡habla!
	\end{tabular}	
	\end{center}
	\bigskip

	en la forma negativa

	\begin{center}
	\begin{tabular}{ l l }
		\textbf{ära tule!}	& ¡no vengas! \\
		\textbf{ära loe!}	& ¡no leas! \\
		\textbf{ära räägi!}	& ¡no hables!
	\end{tabular}	
	\end{center}
	\bigskip

	Nota: El imperativo del verbo lähe/n `ir' se obtiene de otra raíz: \textbf{mine!} `ve', \textbf{ära mine!} `no vayas!'

	\S\ 15. 

	\item como la forma en tiempo presente negativo con la partícula negativa \textbf{ei}, que siempre se coloca delante del verbo. Esta construcción se utiliza para todas las personas en singular y plural.

	\begin{center}
	\begin{tabular}{ l l }
		mina \textbf{ei tule}	& yo no voy a venir \\
		sina \textbf{ei tule}	& tú no vas a venir \\
		tema \textbf{ei tule}	& él/ella no va a venir \\
		& \\
		meie \textbf{ei tule}	& nosotros no vamos a venir \\
		teie \textbf{ei tule}	& ustedes no van a venir \\
		nemad \textbf{ei tule}	& ellos/ellas no van a venir
	\end{tabular}	
	\end{center}	
	\bigskip
\end{itemize}

\S\ 16. La partícula negativa extra \textbf{mitte} puede ser usado para fortalecer el tono de una negación común. 

\begin{center}
\begin{tabular}{ l l }
	ma \textbf{ei tule mitte!} & ¡yo \emph{no} voy a venir!
\end{tabular}	
\end{center}
\bigskip

Observe que la partícula negativa ordinaria \textbf{ei} debe ser incluida.

\S\ 17. Como una forma alternativa para \textbf{ei ole} `no soy/eres/es/somos/son/son/estoy/estas/está/estamos/están/están', la palabra \textbf{pole} se utiliza a menudo, con el mismo significado. (Esta es una contracción de \textbf{ep+ole}, donde \textbf{ep} es una forma arcaica de la partícula negativa \textbf{ei}).

\begin{center}
\begin{tabular}{ l l }
	ta \textbf{ei ole} siin = ta \textbf{pole} siin & `él/ella no está aquí' \\
	ma \textbf{ei ole} valmis = ma \textbf{pole} valmis & `yo no estoy listo' 
\end{tabular}	
\end{center}
\bigskip

\S\ 18. Una respuesta negativa a una pregunta a menudo consiste en la partícula negativa \textbf{ei} con el verbo en cuestión.

\begin{center}
\begin{tabular}{ l l }
	Kas sa \textbf{oled} kodus? & `¿Estás tú en casa?' \\
	\textbf{Ei ole} & `No estoy' \\
	& \\
	Kas te \textbf{tulete}? & `¿Vienen ustedes?' \\
	\textbf{Ei tule} & `No vamos' \\
\end{tabular}	
\end{center}
\bigskip

\S\ 19. El estonio tiene muchos verbos con partículas adverbiales, que corresponden a frases en inglés como `get up', `go out', y similares. Por ejemplo: \textbf{tõusen üles} `me levanto', \textbf{tõusen püsti} `me pongo de pie', \textbf{saan aru} `entiendo' [\emph{lit.}: `adquiero inteligencia'], \textbf{vaatan pealt} `parezco'.\\

En estas situaciones, sólo el verbo cambia en el proceso de conjugación. La partícula acompañante se mantiene sin cambios.

\begin{center}
\begin{tabular}{ l l l l }
	ma \textbf{saan aru} & `yo entiendo' 		& me \textbf{saame aru} `nosotros entendemos' \\
	sa \textbf{saad aru} & `tú entiendes'		& te \textbf{saate aru}	`ustedes entienden' \\
	ta \textbf{saab aru} & `él/ella entiende' 	& nad \textbf{saavad aru} `ellos entienden'
\end{tabular}	
\end{center}
\bigskip

La partícula puede ser separada del verbo por otras partes de la oración. Por ejemplo: ma \textbf{tõusen} kohe \textbf{püsti} `Me levantaré inmediatamente', ma \textbf{saan} hasti \textbf{aru} `Yo entiendo bien'.


% ==================
% 		TEXTO
% ==================
\Large{\section*{Texto}}

Tule siia! Palun, istu. Jutusta, ma kuulan. Räägi kõvasti. Ära räägi nii tasa! Ma ei saa aru, mis sa ütled. Ma ei kuule hästi. Ma kuulen halvasti.\\
Ütle, mis see on! Ma ei tea, mis see on. Vaata, kes seal seisab! Kas sa näed? Ei, ma ei näe. Ma lähen kohe ja vaatan. Mine sinna ja küsi! Tule siia tagasi!\\
Enne mõtle, siis ütle!\\

Kuhu sa lähed, armas sõber? Lähen koju. Perekond on kodus ja ootab. Oota, ma tulen ka kohe! Poeg on kodus. Tütar ei ole. Ta pole veel kodus.\\
Kas te õpite? Ei, me ei õpi. Me lamame ja puhkame. Kas te seisate, või istute? Vend seisab, aga õde istub. Palun vasta, kui ma küsin! Tõuse püsti, kui sa räägid! Kas sa saad aru, mis ma ütlen? Ma kardan, et ma ei saa hästi aru. Kas sa tead, mis seal on? Ma tõesti ei tea.

% =======================
% 		VOCABULARIO
% =======================
\Large{\section*{Vocabulario}}

\begin{tabular}{ l l }
	aru				& entendimiento \\
	ei				& no \\
	enne			& antes, primero \\
	et				& que [\emph{conj.}] \\
	halvasti		& mal \\
	jutusta/n		& (yo) narro \\
	karda/n			& (yo) temo \\
	kodus			& en casa \\
	kohe			& inmediatamente \\
	koju			& a casa \\
	kuhu			& a dónde, adonde \\
	kui				& cuando, si, como \\
	kuula/n			& (yo) escucho \\
	kuule/n			& (yo) oigo \\
	kõvasti			& fuerte, ruidoso \\
	lama/n			& (yo) me reclino \\
	mine			& anda! \\
	mõtle/n			& (yo) pienso \\ 
	nii				& entonces [\emph{adj.}] \\
	oota/n			& (yo) espero \\ 
	perekond		& familia \\
	pole = ei ole	& no soy/eres/es ... \\
	puhka/n			& (yo) descanso \\
	saa/n aru		& (yo) entiendo \\ 
	siia			& hacia aquí \\
	siis 			& entonces \\	
	sinna			& hacia allá \\ 
	sõber			& amigo  \\
	tagasi			& atrás [\emph{adv.}] \\
	tasa			& despacio \\
	tõesti			& verdaderamente \\
	tõusen			& (yo) me levando \\
	tõusen püsti	& (yo) me paro \\
	vaata/n			& (yo) miro \\
	veel			& todavía, más \\
	või				& o \\
	ära				& no \\
	ütle/n			& (yo) digo
\end{tabular}

% ======================
% 		EJERCICIOS
% ======================
\Large{\section*{Ejercicios}}

\begin{enumerate}
	\item \emph{Traduzca al estonio:} Yo esperaré aquí. Tú hablas fuerte, pero él habla despacio. Ellos entienden. Ustedes narran bien. Yo digo que vamos a venir de inmediato. ¿Vendrás de inmediato?\\

	Ellos vienen aquí. Nosotros vamos hacia allá. ¡Ven aquí! ¡Ve allí! Lo veré por mi mismo. Él oye muy bien. ¡Vuelve inmediatamente!. ¡Espera aquí! ¡Habla en voz alta! ¡No hables tan alto! ¿Están de pie o sentados? Di, ¿vas a venir mañana? No, no voy a venir.

	\item \emph{Traduzca al español:} Ma kuulen. Ta kuulab. Sa näed. Me vaatame. Te räägite. Nad ütlevad. Olen kodus. Ta läheb koju. Mina olen siin. Tule ka siia! Nemad on seal. Mine sinna!
\end{enumerate}

% ============================
% 		EXPRESIONES DE ...
% ============================
\Large{\section*{Expresiones de Cortesía}}

\begin{tabular}{ l l }
	\textbf{Kas ma segan?}				& ¿Estoy molestando? \\
	\textbf{Ei, mitte sugugi!}			& No, en absoluto! \\
	\textbf{Pole viga.}					& No hay problema. \\
	\textbf{Ei, sa/te ei sega.}			& No, no me estás molestando. \\
	\textbf{Astu sisse!}				& ¡Adelante! \\
	\textbf{Astuge sisse, palun!}		& ¡Adelante, por favor! \\
	\textbf{Tule siia! Tulge siia!}		& ¡Ven aquí!, ¡Venga aquí! \\
	\textbf{Palun, istu/istuge!}		& Por favor, siéntate/siéntese. \\
	\textbf{Vabanda! Vabandage!}		& ¡Disculpa! ¡Disculpe! \\
	\textbf{Palun vabandust! Vabandust!}& ¡Le pido (su) perdón! ¡Perdón! \\
	\textbf{Vabandage, et tülitan.}		& Perdone por molestar. \\
	\textbf{Palun väga!}				& Por favor (aceptar esto). \\
	\textbf{(Oota) üks silmapilk!}		& (Espere) un momento. \\
	\textbf{Räägi, ma kuulen.}			& Hable, yo escucho. \\
	\textbf{Ära räägi! Kas tõesti?}		& ¡No me digas! ¿En serio?
\end{tabular}

% ======================================
% 		RESPUESTA A LOS EJERCICIOS
% ======================================
\Large{\section*{\Large{Respuesta a los ejercicios}}

\begin{enumerate}
	\item Mina [= ma] ootan siin. Sina räägid kõvasti, aga tema räägib tasa. Nemad saavad aru. Teie jutustate hästi. Ma ütlen, et me tuleme kohe. (Kas sa) tuled kohe?\\

	Nemad tulevad siia. Meie läheme sinna. Tule siia! Mine sinna! Ma näen ise. Ta kuuleb väga hästi. Tule kohe tagasi! Oota siin! Räägi 	kõvasti! Ära räägi nii kõvasti! Kas te seisate või istute? Ütle, kas sa tuled homme? Ei, ma ei tule mitte.

	\item  Yo oigo. Él/ella escucha. Tú ves. Nos miramos. Ustedes hablan. Ellos dicen. Estoy en casa. Él/ella se va a casa. Estoy aquí. ¡Ven hacia aquí también! Ellos están ahí. ¡Ve allí!
\end{enumerate}
%----------------------------------------------------------------------------------------