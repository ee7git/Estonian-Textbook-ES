\newLesson % Lesson 35
\label{lesson-35}

\Grammar % =====================================================================

\Title{The Present (-v) Participle}

% Section 318
\newSection \label{section-318} The present participle is formed by adding the suffix \m{-v} to the stem of the \m{-ma} infinitive. It is used as an adjective, like the English `-ing' form: \m{kasva/v tüdruk} `a growing girl". It may also be translated as a relative clause: `a girl who is growing'. (When English `-ing' words indicate some kind of activity that accompanies the action of a verb, then the gerund or \m{-des} form is employed in Estonian (Lesson 22, \textsection 152.) \\

Here are some examples of how the present participle is formed:

	\twoColumnsTable
	\n{-ma Infinitive} 			& \n{-v Participle} \\
	\m{luge/ma} `to read' 		& \m{luge/v} `reading'  \\
	\m{tööta/ma} `to work' 		& \m{tööta/v} `working'  \\
	\m{võitle/ma} `to struggle' & \m{võitle/v} `struggling'  \\
	\m{lenda/ma} `to fly'		& \m{lenda/v} `flying' \\
	\tableEnd

\m{Töötav inimene} `A working person'. \m{Võitlev rahvas} `A struggling nation'. \m{<<Lendav hollandlane>>} ``The Flying Dutchman''. \m{Kasvav noorus} `Growing (time of) youth'. \m{Rõõmustav uudis} `A gladdening bit of news'. \m{Sobiv juhus} `A fitting occasion'. \\

Note that the names of cases in Estonian employ the present participle: \m{nimetav} kääne [nimeta/ma `to name'] = `\n{naming} [nominative] case', \m{omastav} [omasta/ma `to own'] = `\n{owning} [genitive]', \m{olev} [ole/ma `to be'] = `\n{being} [essive]' and so on.

% Section 319
\newSection \label{section-319} The present participle is used primarily to describe a noun. Like other adjectives, it is declined so as to be in agreement with the noun it describes. The genitive form of a present participle always ends in \m{-a}. For example: \nom \sing \m{töötav} inimene `a working person', \gen \sing \m{töötav/a} inimese `a working person's', \parti \sing \m{töötav/at} inimest, \ill \sing \m{töötav/a/sse} inimesse `into a working person', and so on. \\

\m{Läikivad} silmad `\n{Sparkling} eyes'. \m{Lõhnavad} roosid `\n{Fragrant} roses'. \m{Raskendavad} asjaolud `\n{Trying} circumstances'. \m{Äraootaval} seisukohal `With a wait-and-see attitude [\lit At a \n{waiting} position]'.

% Section 320
\newSection \label{section-320} If the stem of the \m{-ma} infinitive ends in a consonant, for example \m{laul/ma} `to sing', an \m{-e-} is added before the \m{-v} in the nominative singular case for ease of pronunciation: \m{laul/e/v} `singing'. The \m{-e-} is not found in other case forms: \gen \sing \m{laul/v/a}, \parti \sing \m{laul/v/at}, \ill \sing \m{laul/v/asse}, etc. \\

\m{Jooksev} vesi `\n{Running} water'. \m{Jooksval} lindil `On a \n{running} (assembly) line'. \m{Kestev} mõju `A \n{lasting} influence'. \m{Kestvad} lokid `\n{Lasting} curls'. \\

Note: After a short vowel, the final consonant in the stem is doubled before the \m{-ev} suffix is added. For example, \m{nut/ma} `to cry' > \m{nutt/ev} laps `a \n{crying} child', but \m{nut/vad} lapsed `\n{crying} children'. Following the same pattern: \m{kat/ma} `to cover' > \m{katt/ev}, \m{kat/vad}; \m{tap/ma} `to kill' > \m{tapp/ev}, \m{tap/vad}.

% Section 321
\newSection \label{section-321} There is also a passive present participle, ending in \m{-dav} or \m{-tav}. (The passive voice is discussed in the next lesson.) This form often corresponds to adjectives ending in -able or -ible in English. For example: \\

\m{Söö/dav} seen `\n{Edible} mushroom'. Kergesti \m{kättesaa/dav} `Easily \n{reachable}'. Raskesti \m{arusaa/dav} `Hardly \n{understandable}'. \m{Nähtav} enamus `A \n{visible} majority'. \m{Kardetav} vanus `The \n{dangerous} age'. \m{Elukardetav} relv `Lethal weapon [\n{Life-endangering} firearm]'.

\Title{The -v Participle as a Substitute for Clauses}

% Section 322
\newSection \label{section-322} Following verbs that indicate feeling, sensation, or opinion (such as \m{tundma} `to feel', \m{tunduma} `to feel like', \m{paistma} `to seem', \m{nägema} `to see', \m{kuulma} `to hear', \m{leidma} `to find, consider', \m{teadma} `to know', \m{arvama} `to think', \m{lootma} `to hope'), a relative clause can be shortened by dropping the `who', `that', or `which' and exchanging the verb for a present (\m{-v}) participle in the partitive singular (ending in \m{-vat}). \\

The word which served as the subject of the dependent clause thereby becomes an object in the main clause and usually takes the partitive case. Compare the following:

	\threeColumnsTable
	Ma kuulen, \m{et lind laulab}. & = & Ma kuulen \m{lindu} \parti \sing \m{laulvat}. \\
	`I hear \n{that a bird sings}.' & & `I hear a bird singing'. \\
	& & \\
	Ma näen, \m{et mehed tulevad}. & = & Ma näen \m{mehi} \parti \pl \m{tulevat}. \\
	`I see \n{that the men come}.' & & `I see \n{the men coming}.'
	\tableEnd

If the object in the shortened version is a so-called total object, it may be in the genitive case as long as it is singular. (Recall the discussion of case forms of objects in Lesson 28.)

	\threeColumnsTable
	Ma arvan, \m{et vend tuleb}. & = & Ma arvan \m{venna} \gen \sing \m{tulevat}. \\
	`I think \n{that Brother is coming}.' & &
	\tableEnd

% Section 323
\newSection \label{section-323} If the main clause and the dropped dependent clause have the same subject, the reflexive pronoun \m{ise} `-self', \gen \sing \m{enda/enese}, \parti \sing \m{end/ennast} is used in the shortened version.

	\threeColumnsTable
	Ma tunnen, \m{et ma olen} haige. & = & Ma tunnen \m{end} haige \m{olevat}. \\
	`I feel \n{that I am} sick.'	& & `I feel \n{(myself being)} sick.'
	\tableEnd

Note! If an impersonal construction like \m{paistab} `it seems' or \m{tundub} `it feels like' is used in the main clause, the subject of the dependent clause becomes the subject (rather than the object) in the shortened version.

	\threeColumnsTable
	Paistab, \m{et sa oled} haige. & = & \m{Sa} paistad haige \m{olevat}. \\
	`It seems \n{like you are} sick.' & & `\n{You} seem \n{to be} sick.'
	\tableEnd

\Title{The Oblique Mode}

% Section 324
\newSection \label{section-324} The Estonian language has a special verb form which indicates an action or situation of which the speaker only has indirect knowledge. That is, the speaker retells something heard from someone else. This indirect verb form, called \m{kaudne kõneviis} in Estonian, is usually employed after an introductory phrase such as `They say that \dots', `I heard that \dots', `She says that \dots', `It is reported that \dots', etc. Even if the oblique mode is used without an introductory phrase, it is there implicitly, and the English translation should somehow reflect the speaker's lack of direct knowledge. \\

Ma kuulsin, et ta \m{olevat} halb inimene = Ta \m{olevat} halb inimene `I heard that he (\n{reportedly}) is a bad person = (\n{It is said that}) he is a bad person'. Voltaire \m{olevat} ütelnud, et \dots `Voltaire \n{supposedly said} \dots'.

% Section 325
\newSection \label{section-325} The oblique mode is made from the \n{stem of the \m{-ma} infinitive}, by adding the suffix \m{-vat}. For example, \m{aita/ma} `to help' > {\tiny oblique mode} \m{aita/vat}. (This form was originally the partitive singular of the above mentioned present participle, which explains the identical endings.) \\

The oblique mode is the same for all persons, both singular and plural, in both affirmative and negative statements. Here is an example, using the verb \m{tule/ma} `to come':

	\threeColumnsTable
	& 				\n{Affirmative} 		& \n{Negative} \\
	& & \\
	\n{Singular} 	& 1. mina	\m{tule/vat}	& mina	ei	\m{tule/vat} \\
					& 2. sina	\m{tule/vat}	& sina	ei	\m{tule/vat} \\
					& 3. tema	\m{tule/vat}	& tema	ei	\m{tule/vat} \\
	& & \\
	\n{Plural}		& 1. meie	\m{tule/vat}	& meie	ei	\m{tule/vat} \\
					& 2. teie	\m{tule/vat}	& teie	ei	\m{tule/vat} \\
					& 3. nemad 	\m{tule/vat}	& nemad	ei	\m{tule/vat}
	\tableEnd

Tema \m{tulevat} homme, aga teie \m{ei tulevat} `He is supposedly coming tomorrow, but you \pl are reportedly not coming'.

% Section 326
\newSection \label{section-326} The oblique mode has a past perfect tense, which is formed with the auxiliary verb \m{olevat + -nud participle}. \\

Mina \m{olevat tulnud} `I supposedly had come', sina \m{olevat tulnud} `you supposedly had come', etc. \\

As a contraction, sometimes only the \m{-nud} participle is used to express the oblique past tense, especially in stories: \\

\m{Elanud} kord eit ja taat \dots `An old woman and old man were \n{supposed to have lived} once upon a time \dots'. Poisid \m{püüdnud} suure kala The boys (\n{reportedly}) had caught a big fish'.

\Title{The Agent Ending -ja}

% Section 327
\newSection \label{section-327} The suffix \m{-ja} indicates the performer of a verb's action, when added to the stem of the \m{-ma} infinitive. It corresponds to the suffix -er in English (read/er, smok/er) or the phrase `the one who (is dying)'. Here are some examples:

	\twoColumnsTable
	\n{-ma Infinitive} 			& \n{-ja Noun} \\
	\m{mängi/ma} `to play'  	& \m{mängi/ja} `player' \\ 		
	\m{luge/ma} `to read' 		& \m{luge/ja} `reader' \\ 		
	\m{teata/ma} `to announce'  & \m{teata/ja} `announcer' \\ 	
	\m{jooks/ma} `to run'  		& \m{jooks/ja} `runner' \\ 		
	\m{suitseta/ma} `to smoke' 	& \m{suitseta/ja} `smoker'
	\tableEnd

An agent noun ending in \m{-ja} is declined like other nouns. The \n{genitive singular} is always the same as the \n{nominative} (ending in \m{-ja}), the \n{partitive singular} ends in \m{-ja/t}, and the \n{partitive plural} ends in \n{-ja/id}. For example: \m{kirjutaja, -, -t, -id} `writer'.

% Section 328
\newSection \label{section-328} Here are some commonly used examples:

\m{Õpetaja} `teacher' (õpetama `to teach'), \m{laulja} `singer', \m{luuletaja} `poet', \m{näitleja} `actor' (näitlema `to act in a play'), \m{teenija} `servant', \m{ettekandja} `entertainer, (restaurant) waiter', \m{suvitaja} `summer guest', \m{saatja} `sender', \m{saaja} or \m{vastuvõtja} `recipient' (saama `to get', vastu võtma `to receive'), \m{ostja} `buyer, consumer', \m{müüja} `vendor, salesperson', \m{võistleja} `competitor' (võistlema `to compete'), \m{võitleja} `fighter' (võitlema `to fight'), \m{võitja} `winner', \m{kaotaja} `loser', \m{suusataja} `skier', \m{ujuja} `swimmer', \m{võimleja} `gymnast', (kodumaa) \m{kaitsja} `defender (of the homeland)', \m{kohusetäitja} `surrogate, substitute', \m{pilvelõhkuja} `skyscraper' (pilv `cloud', lõhkuma `to wreck'). \\

<<\m{Hüüdja} hääl kõrbe>> ``A (Caller's) Voice in the (Desert) Wilderness''. \m{Lamajat} ei lööda `One does not hit \n{someone who is lying down}'. See \m{näitleja} mängib näidendis esimest \m{armastajat} `That actor plays the first lover in the play'. \m{Võitjale} jääb alati õigus `The winner is always right'.

% Section 329
\newSection \label{section-329} Note! For some verbs which end in \m{-e/ma} in the \m{-ma} infinitive, the \m{-e-} at the end of the stem turns into \m{-i-} before \m{-ja}. For example: \m{tul/e/ma} `to come' > \m{tul/i/ja} `one who comes' \\

Other examples of this type include: olema `to be' > \m{olija} `being, one who is', panema `to put’ > \m{panija} `one who puts', surema `to die' > \m{surija} `dying one', pesema `to wash' > \m{pesija} `washer', nägema `to see' > \m{nägija} `seer', tegema `to make, do' > \m{tegija} `maker, doer'. \\

Töö kiidab \m{tegijat} `The work praises the \n{doer} [the one who does it]'. Kus on \m{tegijaid}, seal on ka \m{nägijaid} `Where there are \n{doers}, there are also \n{watchers} [those who notice it]'. Sa oled mu \m{heategija}! `You are my \n{do-gooder} [benefactor]!'. \m{Koosolijad} plaksutasid käsi `\n{Those in attendance} clapped hands'.

% Section 330
\newSection \label{section-330} The agent (\m{-ja}) noun often accompanies another noun:

\m{Ristija} Johannes `John the \n{Baptist}'. \m{Kaotaja} pool `The \n{losing} side'. \m{Hakkaja} mees `An \n{enterprising} man'. \m{Latsutaja} madu `Rattlesnake [\n{Rattler} snake]'. \\

Compare the agent (\m{-ja}) noun with the present (\m{-v}) participle in the following sentences. Notice the difference in nuance. \\

\m{Haukuja} koer ei ammusta `A dog \n{who barks} (a lot) won't bite'. \m{Haukuv} koer ei ammusta `The dog \n{who is barking} (just now) won't bite'.

\Title{Verbal Nouns Ending in -mine}

% Section 331
\newSection \label{section-331} The suffix \m{-mine}, added to the stem of the \m{-ma} infinitive, is used to form a noun which refers to the action of the verb in general. It corresponds to the -ing suffix in English, when used in a similar context (\eg, Read/ing is fun).

	\twoColumnsTable
	\n{-ma Infinitive} 			& \n{Verbal Noun} \\
	\m{luge/ma} `to read' 		& \m{luge/mine} `reading' \\
	\m{mõtle/ma} `to think' 	& \m{mõtle/mine} `thinking, thought' \\
	\m{õppi/ma} `to study' 		& \m{õppi/mine} `studying' \\
	\m{hinga/ma} `to breathe' 	& \m{hinga/mine} `breathing' \\
	\m{oota/ma} `to wait'		& \m{oota/mine} `waiting'
	\tableEnd

Verbal nouns that end in \m{-mi/ne} in the nominative singular always end in \m{-mi/se} in the genitive singular, in \m{-mi/st} in the partitive singular, and in \m{-mi/si} in the partitive plural. For example: \m{teadmi/ne, -se, -st, -si} `knowing, knowledge'.

% Section 332
\newSection \label{section-332} Note the following common words ending in \m{-mine}: \m{võimlemine} `tumbling, gymnastics' (võimlema `to tumble, do gymnastics'), \m{kuulmine} `(sense of) hearing', \m{nägemine} `(sense of) sight', \m{hääldamine} `pronunciation' (hääldama `to pronounce'), \m{laulmine} `singing', \m{lahkumine} `departure', \m{raiskamine} `waste [wasting]', \m{joomine} `drinking', \m{liigsöömine} `overeating'. \\

\m{Suitsetamine} keelatud! `\n{Smoking} prohibited!'. Jäta \m{suitsetamine} maha! `Quit \n{smoking}!'. Kuidas \m{õppimine} läheb? `How's (your) \n{studying going}?' \m{Teadmised} \pl on võim `\n{Knowledge} \sing is power'. Ei maksa enam \m{rääkimisega} aega viita! `There's no point in wasting (any) more time (with) \n{talking}!'.

% Section 333
\newSection \label{section-333} All the different verb forms derived from the \m{-ma} infinitive have now been presented. These include:

	\enumerateBegin
	\item the simple past or imperfect tense (Lesson 23)
	\item the present (-v) participle (Lesson 35, \textsection 318)
	\item the oblique (-vat) mode (\textsection 325)
	\item the agent (-ja) noun (\textsection 327)
	\item the verbal (-mine) noun (\textsection 331)
	\enumerateEnd

For example:

	\twoColumnsTable
								& \m{mõtle/sin} `I thought'  \\
								& \m{mõtle/v inimene} `a thinking person'  \\
	\n{-ma infinitive} 			& \m{mõtle/vat} `it is thought (that) \dots'  \\
	\m{mõtle/ma} `to think' 	& \m{mõtle/ja} `thinker'  \\
								& \m{mõtle/mine} `thinking (is \dots)'
	\tableEnd

\Text % ========================================================================

\Vocabulary % ==================================================================

\Exercises % ===================================================================

\Expressions % =================================================================

\AnswersToExercises % ==========================================================