\newLesson % Lesson 29

\Grammar % =====================================================================

\Title{Genitive Plural}

The genitive plural is the basis of many other case forms. All the other cases in the plural, except the nominative and partitive, are constructed by adding the appropriate ending to the genitive plural form. (See Lesson 30.)

% Section 228
\newSection The genitive plural has the ending \m{-de} or \m{-te}. Like the genitive singular, it answers the questions \m{kelle?} `whose?' and \m{mille?} `belonging to what?'. The genitive plural has the same uses as the singular (Lessons 7 and 9). For example: \\

\m{Eestlaste} organisatsioonid välismaal `\n{Estonians'} organizations abroad'. See on \m{nende käte} töö `This is the work \n{of their hands}'. \m{Rooside} ja \m{varemete} linn `The town of \n{roses} and \n{ruins} (Visby, Sweden)'. \m{Varjude} riik `The Land \n{of Shadows}'.
Igaüks vastab ise oma \n{sõnade} ja \n{tegude} eest `Everyone is answerable for his own \n{words} and \n{actions}'. \\

The genitive plural is generally derived from the stem of the partitive singular case (Lesson 27).

\Title{The -de Ending}

% Section 229
\newSection a) If the partitive singular ends in a vowel, the genitive plural has the ending \m{-de}, which is added to the partitive singular form. For example:

	\threeColumnsTable
	\n{Nominative singular} & \n{Partitive Singular}	& \n{Genitive Plural} \\
	vend `brother'			& \m{`venda}				& \m{`venda/de} \\
	nimi `name'				& \m{nime}					& \m{nime/de} \\
	riik `state'			& \m{`riiki}				& \m{`riiki/de} \\
	lind `bird'				& \m{`lindu}				& \m{`lindu/de}
	\tableEnd

Compare: kaks \m{`venda} `two brothers' (\part \sing), \m{`venda/de} sõprus `the brothers' friendship' (\gen \pl)

% Section 230
\newSection b) If the partitive singular ends in \m{-d}, the genitive plural has a \m{-de} ending in place of this \m{-d}.

	\threeColumnsTable
	\n{Nominative Singular} & \n{Partitive Singular} 	& \n{Genitive Plural} \\
	maa `earth, ground' 	& \m{maa/d} 				& \m{maa/de} \\
	tee `road, way, tea' 	& \m{tee/d} 				& \m{tee/de} \\
	pea `head' 				& \m{pea/d} 				& \m{pea/de} \\
	hea `good'				& \m{hea/d} 				& \m{hea/de}
	\tableEnd

Compare: mööda \m{tee/d} `along the road' (\part \sing), \m{tee/de} pikkus `the length of the roads' (\gen \pl)

\Title{The -te Ending}

% Section 231
\newSection c) If the partitive singular ends in \m{-t}, this \m{-t} is replaced by the suffix \m{-te} in the genitive plural.

	\threeColumnsTable
	\n{Nominative Singular} & \n{Partitive Singular} 	& \n{Genitive Plural} \\
	raamat `book' 			& \m{raamatu/t} 			& \m{raamatu/te} \\
	inimene `person' 		& \m{inimes/t} 				& \m{inimes/te} \\
	suur `big'				& \m{suur/t} 				& \m{suur/te} \\
	uus `new'				& \m{uu/t} 					& \m{uu/te} \\
	käsi `hand, arm'		& \m{kä/tt} 				& \m{kä/te} (tt>t)
	\tableEnd

Compare: loen \m{raamatu/t} `I am reading a \n{book} (\part \sing), \m{raamatu/te} hinnad `the \n{books'} prices' (\gen \pl)

\Title{Some examples}

% Section 232
\newSection One-syllable words, which contain a diphthong before the final consonant \m{1}, \m{n}, \m{r}, or \m{s}, form the genitive plural with the \m{-te} ending, which is added directly to the nominative singular form. For example:

	\fourColumnsTable
	& \n{Nominative Singular}  	& \n{Partitive Singular} 	& \n{Genitive Plural}  \\
	& \m{koer} `dog'  			& `koer/a 					& \m{koer/te} \\
	& \m{poiss} `boy'  			& `poissi 					& \m{pois/te} (sst>st) \\
	& \m{nael} `nail, pound'  	& `naela 					& \m{nael/te} \\
	& \m{nõel} `needle' 		& `nõela 					& \m{nõel/te} \\
	& \m{õun} `apple'  			& `õuna 					& \m{õun/te} \\
	& \m{sein} `wall' 			& `seina 					& \m{sein/te} \\
	Also: & \m{oks} `branch' 	& `oksa 					& \m{oks/te} \\
	& \m{ots} `end, tip' 		& `otsa 					& \m{ots/te}
	\tableEnd

% Section 233
\newSection Some two-syllable words which end in \m{-t} in the partitive singular have the ending \m{-de} in the genitive plural instead of the expected \m{-te}. For example:

	\threeColumnsTable
	\n{Nominative Singular}  	& \n{Partitive Singular} 	& \n{Genitive Plural} \\
	\m{kõne} `speech'  			& kõne/t 				 	& \m{kõne/de} \\
	\m{aken} `window'  			& aken/t 					& \m{aken/de} \\
	\m{pere} `family' 			& pere/t 					& \m{pere/de} \\
	\m{mure} `worry' 			& mure/t 					& \m{mure/de}
	\tableEnd

This exceptional pattern is also found for \m{hapu} `sour', \m{tubli} `smart', \m{neiu} `maiden', \m{preili} `Miss', \m{proua} `Madam', \m{härra} `Mister', \m{kalju} `cliff, \m{summa} `sum', \m{foto} `photo', \m{firma} `firm, company', \m{auto} `automobile', \m{tütar} `daughter', \m{küünal} `candle', \m{peenar} `flowerbed', and some other, less common words. \\

A similar pattern is also found for words ending in the female suffix \m{-nna}.

	\threeColumnsTable
	\n{Nominative Singular}  			& \n{Partitive Singular} 	& \n{Genitive Plural} \\
	\m{kuninganna} `queen' 				& kuninganna/t 				& \m{kuninganna/de} \\
	\m{lauljanna} `songstress' 			& lauljanna/t 				& \m{lauljanna/de} \\
	\m{sõbranna} `female friend'  		& sõbränna/t 				& \m{sõbränna/de} \\
	\m{ameeriklanna} `American woman'  	& ameerklänna/t 			& \m{ameeriklänna/de} \\
	\m{venelanna} `Russian woman' 		& venelänna/t 				& \m{venelanna/de}
	\tableEnd

% Section 234
\newSection Certain words form the genitive plural by inserting the vowel \m{-e-} between the stem and the \m{-de} ending. In these cases, the \m{-e} is also found in the genitive singular form. There are only six words that show this pattern:

	\fourColumnsTable
	\n{Nominative Singular} & \n{Genitive Singular} & \n{Partitive Singular} 	& \n{Genitive Plural} \\
	meri `sea'  			& \m{mer/e} 			& mer/d 					& \m{mer/e/de} \\
	veri `blood' 			& \m{ver/e} 			& ver/d 					& \m{ver/e/de} \\
	tuli `fire, light'  	& \m{tul/e} 			& tul/d 					& \m{tul/e/de} \\
	uni `dream, sleep'  	& \m{un/e} 				& un/d 						& \m{un/e/de} \\
	lumi `snow'  			& \m{lum/e} 			& lun/d (m>n)  				& \m{lum/e/de} \\
	mõni `some' 			& \m{mõn/e} 			& mõn/d(a) 					& \m{mõn/e/de}
	\tableEnd

% Section 235
\newSection For some words, the genitive singular has a \m{-me} ending, which is missing from both the nominative and partitive singular. In these cases, the genitive plural has a \m{-me/te} ending.

	\fourColumnsTable
	\n{Nominative Singular} & \n{Genitive Singular} & \n{Partitive Singular} 	& \n{Genitive Plural} \\
	võti `key'  			& \m{võt/me} 			& võti/t 					& \m{võt/me/te} \\
	liige `member'  		& \m{Iiik/me} 			& liige/t 					& \m{liik/me/te} \\
	aste `grade, step'  	& \m{ast/me} 			& aste/t 					& \m{ast/me/te} \\
	mitu `several'  		& \m{mit/me} 			& mitu/t 					& \m{mit/me/te} \\
	süda `heart' 			& \m{süda/me} 			& süda/nt (t>nt) 			& \m{siida/me/te}
	\tableEnd

% Section 236
\newSection Some words have a short form ending in \m{-e} in the genitive plural, along with the regular form.

	\twoColumnsTable
	\n{Nominative Singular}  	& \n{Genitive Plural} \\
	jalg `foot, leg'  			& \m{jalge = jalgade} \\
	silm `eye'  				& \m{silme = silmade} \\
	kirjanik `writer'  			& \m{kirjanike = kirjanikkude} \\
	imelik `strange' 			& \m{imelike = imelikkude}
	\tableEnd

The short forms are preferred in certain expressions: \m{jalge all} `underfoot', \m{silme ees} `in front of the eyes'.

\Title{The Interrogatives \n{eks} and \n{ega}}

% Section 237
\newSection The words \m{eks} and \m{ega} are used to begin questions, where one anticipates a definite answer. If the expected answer is affirmative, \m{eks} is used. If the expected answer is negative, \m{ega} is used.

	\twoColumnsTable
	\m{Eks} ole tõsi?			& `It is true, right?' \\
	\m{Ega} sa kauaks jää? 		& `You won't stay long, will you \sing?'
	\tableEnd

% Section 238
\newSection Note that the verb in a sentence begun with \m{eks} or \m{ega} is in a negativized form. That is, the verb is the same (without a personal ending) as it would be in a negative statement with the word \m{ei} `not'. In other words, \m{eks} or \m{ega} could be replaced by \m{kas\dots ei}, without affecting the rest of the sentence.

	\twoColumnsTable
	\m{Kas} sa \m{ei} tule homme?  	& `Aren't you \sing coming tomorrow?'  \\
	\m{Eks} sa tule homme?  		& `You are coming tomorrow, aren't you?'  \\
	\m{Ega} sa tule homme? 			& `You aren't coming tomorrow, are you?'
	\tableEnd

Note: In modem usage, the affirmative \m{eks} is often followed by the verb conjugated to fit the subject (that is, with a personal ending), while the negative \m{ega} may be accompanied by a redundant \m{ei}: \m{Eks} sa tuled homme? \m{Ega} sa \m{ei} tule homme?

% Section 239
\newSection \m{Eks} and \m{ega} are often used as \n{emphatic} particles in sentences, to indicate certainty about something. \m{Eks} is used in affirmative statements, \m{ega} in negative ones.

\m{Eks} sa ise tea(d) kõige paremini `You \sing surely know best yourself'. \m{Eks} sa ise näe(d) `You'll see for yourself'. \m{Eks} ma öelnud, et täna tuleb külm ilm `I said that there would be cold weather today'. \\

\m{Ega} ta ei maga `He's not sleeping'. Ära karda, \m{ega} ta sulle midagi (ei) tee `Don't be afraid, he won't do anything to you \sing'. \m{Ega} ma seda meelega (ei) teinud `I didn't do it on purpose'. \\

\m{Ega} may also be used to start a sentence where one wants to request something in a tentative or restrained manner, even if the expected answer is positive: \m{Ega} sa ei saa mulle raha laenata? `You \sing wouldn't be able to loan me money, would you?'.

% Section 240
\newSection The combination \m{ei\dots ega} means `neither\dots nor'.

\m{Ei} seda \m{ega} teist `Neither (this) one nor the other'. Ta \m{ei} söö \m{ega} joo palju `She neither eats nor drinks much'. Mul pole [\m{ei} ole] tuju \m{ega} tahtmist seda teha `I have neither an inclination nor a desire to do it'.

\Text % ========================================================================

\Vocabulary % ==================================================================

\Exercises % ===================================================================

\Expressions % =================================================================

\AnswersToExercises % ==========================================================