\newLesson % Lesson 40

\Grammar % =====================================================================

\Title{Review of Prepositions and Postpositions \\ \bigskip
I. Prepositions}

% Section 398
\newSection \n{With the genitive case:} \\

\m{läbi} `through' -- läbi tule ja vee `through fire and water'. \\

\m{peale} `besides' -- peale selle `besides this', peale kauba `into the bargain (besides the merchandise)'.\\

\m{üle} `over' -- Ta kasvab mul üle pea `He is growing over [taller than] my head'. Me sõitsime üle silla `We rode over the bridge'. \\

\m{ümber} `around' -- ümber tule `around the fire', käib nagu kass ümber palava pudru `walks like a cat around hot porridge'.

% Section 399
\newSection \n{With the partitive case:} \\

\m{alla} `down' -- alla mäge `down the mountain(side)'. \\

\m{enne} `before' -- enne lõunat (e.l.) `before noon (a.m.)', enne Kristust (e.Kr.) `before Christ (B.C.)'. \\

\m{kesk} or \m{keset} `amid, in the middle of' -- kesk `linna `in the middle of town', kesk `päeva `in the middle of the day'. \\

\m{mööda} `along' -- mööda teed `along the road', mööda tänavat `along the street'. \\

\m{piki} `along(side)' -- piki kallast `along the shore', piki `kraavi `alongside the ditch'. \\

\m{pärast} or \m{peale} `after' -- pärast [peale] lõunat (p.l.) `after noon (p.m.)', pärast Kristust (p.Kr.) `after Christ' (A.D.)', pärast `surma `after death'. \\

\m{vastu} `against' -- vastu `seina `against the wall', vastu `voolu `against the current'.

% Section 400
\newSection \n{With other cases:} \\

\m{ilma} `without' (with abessive case)’ -- ilma minuta `without me', ilma rahata `without money'. \\

\m{kuni} `until, up to, as far as' (with terminative case) -- kuni linnani `as far as the town', kuni pühapäevani `until Sunday'. \\

\m{ühes} or \m{koos} `with, together with' (with comitative case) -- ühes vennaga `together with Brother', koos teistega `with others'.

\Title{II. Postpositions}

% Section 401
\newSection \n{With the genitive case:} \\

\m{alla - all - allt} `(to) under - underneath - from under' -- (See \textsection 107 in Lesson 17.) Pole midagi uut päikese all `There is nothing new under the sun'. \\

\m{ette - ees - eest} `(to) in front of - in front of - from in front of' -- (See \textsection 107 in Lesson 17.) Also note: \m{eest} `for'. Võitlus olemasolu eest `Struggle for existence'. Iga hinna eest `For [At] any price'. Ta põgeneb vaenlase eest `He flees before the enemy'. Tee seda minu eest `Do it for me (in my place)'. Viie dollari eest `for five dollars'. Üks kõigi eest ja kõik ühe eest `One for all and all for one'. \\

\m{jaoks} `for' -- Mul on midagi sinu jaoks `I have something for you \sing'. \\

% ISSUE '--' bad allocated
\m{juurde - juures - juurest} `to - at - from' -- Ma sõidan venna juurde `I (will) drive to Brother's (place)'. Elan venna juures `I am staying at my brother's (place)'. Sõidan ära venna juurest `I am leaving (from) my brother's (place)'. Note: hea tervise juures `in good health'. \\

\m{järele - järel - järelt} `(to) after - after - from after' -- Ma tulen raamatu järele `I will come after [to fetch] the book'. Ta käib minu järel `She is walking after [behind] me'. Ta kadus minu järelt `He disappeared (from) behind me'. Note: Üks õnnetus teise järele `One misfortune after another'. Ma igatsen sinu järele `I long for [after] you'. \\

\m{järgi} `after, according to, following' -- Avakõne järgi tuleb kontsert `After the opening speech comes the concert'. Tegin tema õpetuse [nõuande] järgi `I did it according to her instruction [advice]'. Minu kella järgi olete hiljaks jäänud `According to my watch, you \pl are late'. Koon sukki vana mustri järgi `I knit socks, following an old pattern'. \\

\m{kaudu} `via, through, by means of' -- Ta sõitis Taani Malmö kaudu `He went to Denmark via Malmö'. Posti kaudu `By mail'. \\

% ISSUE '--' bad allocated
\m{keskele - keskel - keskelt} `(in)to the middle - in the middle - from the middle' -- Mine toa keskele! `Go to the middle of the room' (whither?). Ta seisis toa keskel `He stood in the middle of the room' (where?). Linna keskelt `from the middle of town' (whence?). Note: inimeste keskel `among people'. \\

% ISSUE '--' bad allocated
\m{kohale - kohal - kohalt} `(to) above - at, above - from above' -- Lennuk ilmus linna kohale `An airplane appeared above the city'. Lennuk lendab linna kohal `An airplane flies above the city'. Lennuk kadus linna kohalt `An airplane disappeared from above the city'. \\

\m{kohta} `about, regarding' -- Kelle kohta see käib? `Whom is that about? [To whom does that apply?]'. Teated tema kohta `Messages about him'. Oma vanuse kohta on ta suur `For [In terms of] her age she is big'. \\

\m{kätte - käes - käest} `to [into the hand of] - with [in the hand of] - from [out of the hand of]' -- Anna raamat minu kätte `(You \sing) Give the book to me'. Raamat on minu käes `The book is in my possession [in my hand]'. Sain raamatu isa käest [isalt] `I got the book from Father'. Note: Koht päikese käes `A place in the sun'. \\

\m{peale - peal - pealt} `onto - on - off -- (See \textsection 105 in Lesson 17.) Pane raamat laua peale [lauale] `(You \sing) Put the book onto the table'. Vaas on laua peal [laual] `The vase is on the table'. Võta lilled laua pealt [laualt] `Take the flowers off the table'. \\

\m{pärast} `after, on account of, for' -- ühe aasta pärast `after one year', sinu pärast `on mi account of you \sing', sellepärast `for this reason'. \\

\m{sisse - sees - seest} `into - in - out of' -- (See \textsection 105 in Lesson 17.) Note: selle aja sees `during this time'. \\

\m{taha - taga - tagant} `(to) behind - behind - (from) behind' -- Ma panin käed selja taha `I put my hands behind my back' (whither?). Ta seisab ukse taga `He stands behind the door' (where?). Tule ukse tagant välja `(You \sing) Come out from behind the door' (whence?). Note: Ta tõusis laua tagant `She rose from (behind) the (work/writing) table'. But: Ta tõusis lauast `She rose from the (dining) table'. \\

\m{vahele - vahel - vahelt} `(to) between - between - (from) between' -- Tule istu meie vahele! `(You \sing) Come sit between us!' Maa ja taeva vahel `Between earth and heaven'. Päike tuli pilvede vahelt välja `The sun came out from between the clouds'. Note: Ma läksin kolonnide vahelt läbi `I went through (between) the columns'. \\

\m{vastu} `against, toward' -- Mul pole midagi selle vastu `I have nothing against it'. Note: Ma tunnen huvi spordi vastu `I am interested in [feel interest toward] sports'. Huvi kunsti vastu `An interest in art'. \\

\m{äärde - ääres - äärest} `(to) beside - beside - (from) beside' -- (See \textsection 107 in Lesson 17.) Lähen laua äärde `I go up to (beside) the table'. Istume mere ääres `We sit by the sea'. Tuleme mere äärest `We come from the seaside [from beside the sea]'. \\

\m{üle} `over, about' -- Räägime selle üle [sellest] homme `Let's talk about this tomorrow'. Nad olid tema tuleku üle väga imestunud `They were very surprised by [over] his arrival'. \\

\m{ümber} `around' -- Õpilased olid kogunenud õpetaja ümber `The students had gathered around the teacher'.

% Section 402
\newSection \n{With other cases:} \\

\m{mööda} `along' (with partitive case) -- teed mööda [= mööda teed] `along the way'. \\

\m{alla} `down' (with elative case) -- Ta tuli trepist alla `He came down the stairway'. \\

\m{läbi} `through' (with elative case) -- kanalist läbi `through the canal'. \\

\m{peale} `since' (with elative case) -- Ma olen töötanud siin hommikust peale `I have been working here since morning'. \\

\m{saadik} `since' (with elative case) -- suvest saadik `since last summer', mis ajast saadik? `since when?'. Note: põlvest saadik `up to the knee'.

\Title{Cases Governed by Verbs and Adjectives}

% Section 403
\newSection Verbs and adjectives, like prepositions and postpositions, require certain case forms for nouns and pronouns used with them. For example, \m{olenema} kellest? millest? `to depend on whom? on what?' calls for the elative case with the \m{-st} suffix. (See oleneb \m{asjaoludest} `That depends \n{on the circumstances}'.) \\

In some instances, it is possible to use several different case forms after the same word, \eg, jagama mida? millega? `to divide what? by/with what?'. (Jaga \m{kaksteist neljaga} `Divide \n{twelve by four}'.) \\

The section below reviews the most important instances where verbs and adjectives govern case forms. See also the relevant material in Lessons 14 (\textsection 83), 15 (\textsection 90), 18 (\textsection 111-116), 25 (\textsection 179), and 28 (\textsection 225), and recall the ways in which verbs and adjectives govern the form of \m{-ma} and \m{-da} infinitives (Lessons 21-22, \textsection 131-151).

% Section 404
\newSection \n{Verbs:} \\

\m{aitama} keda? mida? `to help': Aita \m{mind}! `Help me!'. \\

\m{aitab} {\small [\n{impersonal}]} millest? `suffices': \m{Sellest} aitab `That's enough (of that)'. \\

\m{armuma} kellesse? millesse? `to fall in love': Poiss armus \m{tüdrukusse} `The boy fell in love with the girl'. \\

\m{aru saama} kellest? millest? `to understand': Ma ei saa \m{sellest sõnast} aru `I do not understand this word'. \\

\m{arvestama} keda? mida? (kellega? millega? in informal speech) `to reckon with': Arvesta \m{seda võimalust}, et \dots `Reckon with the possibility that \dots'. \\

\m{eelistama} keda? kellele? mida? millele? `to prefer': Ma eelistan \m{teatrit kinole} `I prefer the theater over the cinema'. \\

\m{harjuma} kellega? millega? `to get accustomed to': Inimene harjub \m{kõigega} `A person gets used to everything'. \\

\m{huvi tundma} kelle vastu? mille vastu? `to be interested in': Ma tunnen huvi \m{selle töö} vastu `I am interested in this work'. \\

\m{imestama} keda? mida? mille üle? `to be amazed': Ma imestan su \m{kannatlikkust} `I am amazed by your \sing patience'. \m{Mille üle} sa imestad? `What amazes you \sing?'. \\

\m{imetlema} keda? mida? `to adore': Nad imetlevad \m{moodsat kunsti} `They adore modem art'. \\

\m{jagama} mida? millega? `to divide': Jaga \m{kaksteist neljaga} `(You \sing) Divide twelve by four'. \\

\m{jätkama} mida? `to continue': Me jätkame oma \m{tööd} `We continue our work'. \\

\m{jätkub} millest? `to suffice': \m{Sellest} jätkub [= aitab] `That is enough (of that)'. \\

\m{kaasa tundma} kellele? millele? milles? `to sympathize': Ma tunnen \m{teile} kaasa teie \m{leinas} `I sympathize with you \pl in your grief'. \\

\m{kadestama} keda? `to envy': Ta kadestab \m{sind} `He envies you'. \\

\m{kaebama} kellele? mille üle? `to complain': Ma kaeban \m{sulle} oma \m{häda üle} `I complain to you \sing about my trouble'.

\m{kahtlema} kelles? milles? `to doubt': Ma kahtlen tema m{sõpruses} `I doubt (the sincerity of) her friendship'. \\

\m{keelduma} millest? `to refuse, say no to': Ta keeldus \m{sellest} `He refused (to go along with) it'. \\

\m{kiinduma} kellesse? millesse? `to become attached or devoted to': Õpilane kiindus oma \m{õpetajasse} `The student became attached to her teacher'. \\

\m{kohanema} kellega? millega? `to adjust, adapt to': Me kohanesime \m{ümbrusega} `We adjusted to the surroundings'. \\

\m{kohtama} keda? `to meet': Ma kohtasin \m{sõpra} tänaval `I met my friend on the street'. \\

\m{kohtlema} keda? `to treat, handle': See õpetaja kohtleb oma \m{õpilasi} sõbralikult `This teacher treats his pupils in a friendly manner'. \\

\m{kuulama} keda? mida? `to listen': Kuula \m{teda}! `(You \sing) Listen to her!'. Note: Kuula \m{sõna}! `Obey! [\lit Listen to (my) word!]'. \\

\m{kuuluma} kellele? millele? `to belong to': Raamat kuulub \m{poisile} `The book belongs to the boy'. \\

\m{kõnelema} kellest? millest? kelle üle? mille üle? `to speak': Me kõneleme \m{muusikast} `We speak of music'. \\

\m{küsima} kellelt? mida? `to ask': Laps küsis \m{isalt}, kas \dots `The child asked the father, whether \dots'. Me küsisime \m{õpetajalt nõu} `We asked the teacher for advice'. \\

\m{loobuma} kellest? millest? `to decline, renounce': Ma loobun \m{sellest aust} `I decline this honor'. \\

\m{lootma} kellele? millele? `to count on, put hopes in': Me kõik loodame \m{sinule} `We are all counting on you \sing'. \\

\m{lugu pidama} kellest? millest? `to respect': Ma pean \m{sinust} lugu `I respect you \sing'. \\

\m{läbi nägema} kellest? millest? `to see through': Ma näen \m{sust} [sinust] läbi `I see through you \sing [I see you for what you are]'. \\

\m{mõju avaldama} kellele? millele? `to influence, make an impression on': Viimane kõne avaldas \m{rahvale} suurt mõju `The last speech made a big impression on the people'. \\

\m{naerma} kelle üle? mille üle? `to laugh at, smile about': Kõik naersid \m{nalja üle} `Everyone laughed at the joke'. \\

\m{olenema} kellest? millest? `to depend on': See oleneb \m{asjaoludest} `That depends on the circumstances'. \\

\m{omama} mida? `to own, have': See küsimus omab \m{suurt tähtsust} `This question has great importance'. \\

\m{osa võtma} millest? `to take part in': Me võtsime \m{võistlusest} osa `We took part in the contest'. \\

\m{oskama} mida? `to know, be able': Kas te oskate eesti \m{keelt} `Do you \pl know (how to speak) Estonian?'. \\

\m{pettuma} kelles? milles? `to be disappointed in': Nad pettusid näitleja \m{mängus} `They were disappointed in the actor's playing (of his role)'. \\

\m{puudutama} keda? mida? `to touch': Ära puuduta \m{seda lille}! `Don't (you \sing) touch this flower!'. \\

\m{puutuma} kellesse? millesse? `to concern': See ei puutu \m{minusse} `This does not concern me'. \\

\m{rääkima} kellest? millest? kellega? `to talk': Me räägime \m{maalist kunstnikuga} `We talk about the painting with the artist'. \\

\m{sarnanema} or \m{sarnlema} kellega? millega? `to resemble, be like': Poeg sarnaneb \m{isaga} `The son resembles the father'. \\

\m{suhtuma} kellesse? millesse? kellele? millele? `to relate to, have an attitude toward': Kuidas te suhtute \m{sellesse küsimusse}? `How do you \pl feel about [stand on] this question?'. \\

\m{tegelema} millega? `to deal, be occupied with': Kas te tegelete \m{poliitikaga}? `Are you \pl in [occupied with] politics?'. \\

\m{tutvuma}, \m{tutvunema}, or \m{tuttavaks saama} kellega? millega? `to become acquainted with': Ma tutvusin \m{temaga} [Ma sain \m{temaga} tuttavaks] alles eile `I became acquainted with him just yesterday'. \\

\m{tutvustama} keda? mida? kellele? `to introduce': Ma tutvustan sind \m{selle härraga} `I shall introduce you to this gentleman'. \\

\m{tänama} keda? mille eest? `to thank': Ma tänan \m{teid kingituse eest} `I thank you \pl for the gift'. \\

\m{uskuma} keda? mida? kellesse? millesse? `to believe (in)': Ma usun \m{sind} `I believe you \sing'. Ta usub \m{Jumalasse} `She believes in God'. \\

\m{valdama} mida? `to have command of': Õpilane valdab eesti \m{keelt} `The student has command of the Estonian language'. \\

\m{vastama} kellele? millele? `to (cor)respond': See ei vasta \m{tõele} `It is not true [It does not correspond to the truth]'. \\

\m{veenduma} milles? `to be convinced of': Ma olen veendunud sinu \m{aususes} `I am convinced of your \sing honesty'. \\

\m{võlgnema} kellele? mida? `to owe': Ma võlgnen \m{sulle palju tänu} `I owe you \sing many thanks'. \\

% Section 405
\newSection \n{Adjectives:} \\

\m{kade} kellele? (kelle peale?) `jealous': Miks sa \m{mulle} kade oled? `Why are you \sing jealous of me?'.\\

\m{kasulik} kellele? millele? `useful': Võimlemine on \m{tervisele} kasulik `Exercise is good [useful] for health'. \\

\m{kindel} kelles? milles? `sure': Ta on oma \m{võidus} kindel `He is sure of his victory'. Note: Ma olen sinule [sinu peale] kindel `I am sure about you \sing'. \\

\m{pahane} kellele [kelle peale]? millele? mille üle? mille pärast? `angry, mad': Ära ole \m{mulle} [\m{minu peale}] \m{selle pärast} pahane `Don't (you \sing) be mad at me about that'. \\

\m{rõõmus} kelle üle? mille üle? `happy': Ma olen rõõmus \m{kingituse üle} `I am happy about the gift'. \\

\m{sarnane} kellega? millega? `similar, like': Ta on emaga \m{sarnane} `She is like her mother'. \\

\m{teadlik} millest? `aware': Ma ei olnud \m{sellest} teadlik `I was not aware of this'. \\

\m{tänulik} kellele? millele? mille eest? `grateful': Me oleme \m{sulle} väga tänulikud \m{abi} eest `We are very grateful to you \sing for the help'. \\

\m{vihane} [= pahane] kellele [kelle peale]? millele? mille üle? mille pärast? `angry': Ära ole \m{mulle} [\m{minu peale}] vihane `Don't (you \sing) be mad at me about that'.

\Title{Conjunctions}

% Section 406
\newSection Conjunctions serve to connect clauses or phrases.

	\twoColumnsTable
	\m{aga [= kuid]} `but, however'  									& \m{mitte ainult \dots vaid ka \dots} `not only \dots but also \dots' \\
	\m{ehk} `or, that is' (Exercise 29:5)  						& \m{nagu} `as' \\
	\m{ehkki [= kuigi]} `although'  									& \m{nii et} `so that' \\
	\m{enne kui} `before'  														& \m{niikaua kui} `as long as' \\
	\m{ei \dots ega \dots} `neither \dots nor \dots'  & \m{nii \dots kui (ka) \dots} `both \dots and \dots' \\
	\m{et} `that' 																		& \m{niipea kui} `as soon as' \\
	\m{ja [= ning]} `and' 														& \m{seega [= niisiis]} `thus' \\
	\m{juhul kui} `in case' 													& \m{sel ajal kui} `while' \\
	\m{kas} `if, whether' 														& \m{sellepärast et} `because' \\
	\m{kas \dots või \dots} `either \dots or \dots' 	& \m{sest} `for' \\
	\m{kui} `if, when, as' 														& \m{vaid} `rather' \\
	\m{kuigi} `though' 																& \m{või} `or' (Exercise 29:5, p. 201) \\
	\m{kuna} `since, (inasmuch) as, while'						& 
	\tableEnd

% Section 407
\newSection Note the use of the following conjunctions: \\

Ma ei tea, \m{kas} ta tuleb \m{või} ei tule `I don't know \n{whether} he is coming \n{or} not coming'. \\

Vend on sama vana \m{kui} õde `The brother is (just) as old \n{as} the sister'. Ma olen vanem \m{kui} sina `I am older \n{than} you \sing'. \m{Kui} sa tuled siia, võta raamat kaasa `When you \sing come here, take the book along'. Tee seda, \m{kui} sa oskad `(You \sing) Do this, if
you can'. \\

\m{Kuna} oli juba hilja, hakkasime koju minema `\n{Since} it was already late, we started to go home'. Jüri istus toas ja luges, \m{kuna} teised poisid mängisid väljas jalgpalli `Jüri sat indoors (in the room) and read, \n{while} the other boys played football outside'.

\Title{Common Abbreviations}

% Section 408
\newSection

	\oneColumnTable
	\m{a.} 								= aastal `(in the) year': 1994. a. `in the year 1994' \\
	\m{e.} 								= ehk `or, that is' \\
	\m{e.l.} 							= enne lõunat `a.m., in the morning' \\
	\m{hr.} 							= härra `Mister, Mr.' \\
	\m{jm.} 							= ja muud `and more; and other things' \\
	\m{jms.} 							= ja muud seesugust `and (more) such; and more of the same' \\
	\m{j.} 								= järgi `after, according to' \\
	\m{jne.} 							= ja nii edasi `and so on; etc.' \\
	\m{jt.} 							= ja teised `and others; et al.' \\
	\m{kl.} 							= kell `o'clock'; klass `class' \\
	\m{lk.} 							= lehekülg `page [p.]' \\
	\m{Ip.} or \m{Igp.} 	= lugupeetud `respected' \\
	\m{m.a.}	 						= möödunud aastal `last year' \\
	\m{nn.} 							= niinimetatud or nõndanimetatud `so-called' 
	\m{nr.} 							= number `number [no.]' 
	\m{näit.} 						= näiteks `for example; \eg' \\
	\m{n.ö.} 							= nii-öelda `so to speak' \\
	\m{p.l.} 							= peale lõunat, pärast lõunat `p.m., in the aftemoon/evening' 
	\m{pr.} 							= proua `Madam, Mrs.' 
	\m{prl.} 							= preili `Miss' \\
	\m{s.a.} 							= sel aastal `this year': 23. augustil s.a. `on August 23 of this year' \\
	\m{skp.} 							= selle kuu päeval `of this month': 25. skp. `on the 25th of this month' \\
	\m{s.o.} 							= see on `that is to say; \ie' \\
	\m{s.t.} 							= see tähendab `that means' \\
	\m{tel.} 							= telefon `telephone' \\
	\m{v.a.} 							= väga austatud `very honored' \\
	\m{vrd.} 							= võrdle `compare' \\
	\m{vt.} 							= vaata `see' \\
	\tableEnd

Without periods: international units of measurement, mathematical terms--\m{m} meeter, \m{g} gramm, \m{kg} kilogramm, \m{t} tonn, \m{l} liiter, \m{log} logaritm, etc.

\Text % ========================================================================

\Vocabulary % ==================================================================

\Exercises % ===================================================================

\Expressions % =================================================================

\AnswersToExercises % ==========================================================