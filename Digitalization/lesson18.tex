
\newLesson % Lesson 18

\Grammar % =====================================================================

\Title{Inner or Outer Locative Cases?}

The inner locative cases (ending in \m{-sse}, \m{-s}, \m{-st}) generally correspond to prepositional phrases using `in' in English, while the outer locative cases (\m{-le}, \m{-1}, \m{-It}) correspond to expressions with the preposition `on'. For example: \m {toas} `in the room', \m{laual} `on the table'.

% Section 109
\newSection There are, however, some instances where inner locative cases are used in Estonian but not translated as prepositional phrases with `into, in, out of'. Instead, the English translation may employ prepositions such as `to, on/at, from', which are normally the counterparts of outer locative cases: \\

Läheme \m{teatrisse, kinno (kinosse), ooperisse, bussi} `Let's go to the theater, to the cinema, to the opera, on the bus'.

Mis \m{tänavas} [= tänaval] te elate? `On what street do you \pl live?'.

Ma töötan \m{haiglas, raamatukogus, muuseumis} `I work at the hospital, at the library, at the museum'.

Me viibime \m{jaamas, kohvikus, külas, pulmas, rannas} `We linger at the station, at the café, on a visit, at a wedding, at the shore'.

Nad õpivad \m{ülikoolis, instituudis} `They study at the university, at the institute'.

Pane müts \m{pähe}! `Put your hat on!'.

% Section 110
\newSection On the other hand, the outer locative cases may be used in Estonian, where the English translation employs `into, in, out of':

	\oneColumnTable
	Istu \m{toolile} `(You \sing Sit in the chair'. \\
	Pane lilled \m{aknale} `Put the flowers in the window'. \\
	Ta on \m{kolmandal kohal} `He is in third place'. \\
	Kõik asjad on \m{kohal} `Everything is in place'.
	\tableEnd

\Title{Movement or Stationary Location?}

% Section 111
\newSection It may seem strange to a non-native speaker that Estonian uses the illative (\m{-sse}) and allative (\m{-le}), which answer the question `where to? whither?', to indicate the place where something remains or is left: \\

Ta \n{jääb} \m{haiglasse} `He is staying in the hospital [\lit to the hospital]'. \n{Jäta} raamat \m{lauale}! `(You \sing) Leave the book on(to) the table'. Ma \n{unustasin} raha \m{koju} `I left my money at [to] home'. See väike raamat \n{mahub} \m{taskusse} `This little book fits in(to) the pocket'.

% Section 112
\newSection Here are some other examples of verbs that indicate movement toward something and thus require cases that indicate direction: \\

\n{Kirjutan} \m{tahvlile, vihikusse} `I write on(to) the blackboard, in(to) the notebook'. \n{Koputan} \m{uksele} `I knock on(to) the door'. \n{Vajutan} \m{nupule} `I push (onto) the button'. \n{Toetun} \m{kepile} `I lean on(to) the cane'. \n{Riputan} pildi \m{seinale} `I hang the picture on(to) the wall'. \n{Suudlen} \m{suule} `I kiss on(to) the lips [mouth]'. \n{Haigestuti} \m{grippi} `I am getting sick with [into] the flu'. \n{Sureb} \m{tiisikusse} [= tiisikuse kätte] `He is dying of [into] tuberculosis'. \n{Peidan} end \m{puu taha} `I hide (to) behind the tree'.

% Section 113
\newSection The elative (\m{-st}) and ablative (\m{-lt}) cases, which answer the question `where from? whence?', are used to indicate: \\

	\oneColumnTable
	a) the place where you seek, find, or buy something \\
	\tableEnd

Mis sa \m{maast} leidsid? `What did you find on [out of] the ground?'. Otsi \m{põrandalt}! `(You \sing) Search (off) the floor!'. Osta \m{poest} üks raamat `Buy a book from [out of] the store'. \m{Kust} sa selle ostsid? `Where did you \sing buy this (from)?'

% Section 114
\newSection b) the book, newspaper, radio, etc. where you read, see, or hear something \\

Ma \n{lugesin} seda \m{raamatust, lehest}, sinu ºm{kirjast} `I read it in [out of] a book, in the newspaper, in your letter'. Ma \n{näen} su \m{pilgust}, et sa oled vihane `I see from [out of] your look that you are mad'. Ma \n{kuulan} päevauudiseid \m{raadiost} `I am listening to the news on [out of] the radio'. 

% Section 115
\newSection c) the way or the opening through which some movement occurs \\

Lähen \m{trepist} üles `I go up (out of) the stairs'. Ronin \m{redelist} alla `I climb down (out of) the ladder'. Varas tuli \m{aknast} sisse `The thief came in through [out of] the window'. \\

Note also: Rong möödub \m{jaamast} `The train passes (out of) the station'. Ta läheb \m{minust} mööda ja ei tereta `He goes past me and does not greet me'. Inimene saab üle \m{igast raskusest} `A person overcomes [gets over out of] every difficulty'.

% Section 116
\newSection  d) the object one is holding or grabbing \\

\n{Hoian} sind \m{käest, sõrmest, kõrvast} `I hold you by [out of] the hand, by the finger, by the ear'. \n{Võta} mul \m{käest} kinni `(You \sing) Take hold of my hand'. Koer \n{hammustas} mind \m{jalast} `The dog bit me in [out of] the leg'. Politseinik \n{haaras} vargal \m{kraest} kinni `The policeman grabbed the thief by the collar'.

\Text % ========================================================================

\Vocabulary % ==================================================================

\Exercises % ===================================================================

\Expressions % =================================================================

\AnswersToExercises % ==========================================================