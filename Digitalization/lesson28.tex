\newLesson % Lesson 28

\Grammar % =====================================================================

\Title{The Direct Object}

% Section 215
\newSection The direct object of a verb is the thing which receives the action of the verb, as in `I read the \n{newspaper}. The dog buried a \n{bone}. She will rent a \n{car}.'. In English, the direct object takes the same form as the word would when used as the subject of a sentence, except for most pronouns: `I know \n{them}. They like \n{us}. We elected \n{her}. \n{Whom} is she addressing? Who called \n{me} a moment ago?'. \\

In Estonian, the rules for the direct object are much more complex. Depending on the circumstances, the object may take on any of three different case forms: nominative, genitive, or partitive. For example:

	\twoColumnsTable
	Võta \m{raamat}! \small{(nominative)} 		& (You \sing) Take the \n{book}! \\
	Ma võtan \m{raamatu}. \small{(genitive)}  &	I (will) take the \n{book}. \\
	Ma loen \m{raamatut}. \small{(partitive)} &	I am reading the \n{book}.
	\tableEnd

In order to be able to use the right case form in the right situation, you should realize that the direct object (called \m{sihitis} in Estonian) can be considered either as a ``total'' object or a ``partial'' object.

\Title{The Total Object}

% Section 216
\newSection The total object (\m{täissihitis} in Estonian) may also be called the definite, complete, or whole object. It is used in sentences where all three of the following conditions are met:

	\begin{enumerate}
	\item The sentence is affirmative.
	\item The action of the verb leads to completion.
	\item The object is affected in its entirety.
	\end{enumerate}

% Section 217
\newSection In an imperative sentence (where a command is given), the total object is in the nominative case. In a declarative sentence (where a statement of fact is given), the total object is in the genitive case if it is in the singular and in the nominative case if it is in the plural.

\begin{center}
The Total Object in Affirmative Sentences
\end{center}

	\twoColumnsTable
	Commands												 					& Declarations \\
	& \\	
	Võta \m{raamat}! \small{\nom \sing} 	 	& Ma võtsin \m{raamatu}. \small{\gen \sing} \\
	`(You \sing) Take the \n{book}!' 					& `I took the \n{book}.' \\
	& \\
	Võta \m{raamatud}! \small{\nom \pl} 	 	& Ma võtsin \m{raamatud}, \small{\nom \pl} \\
	`(You \sing) Take the books!'		 					& `I took the \n{books}.'
	\tableEnd

\Title{The Partial Object}

% Section 218
\newSection The partial object (called \m{osasihitis} in Estonian) occurs when any one of the following conditions is met:

	\begin{enumerate}
	\item The sentence is negative.
	\item The verb expresses an on-going, incomplete, or repeated action.
	\item The action of the verb is directed toward an undetermined or indefinite portion of the object.
	\end{enumerate}

The partial object is always in the partitive case, whether a singular or plural noun is involved

% Section 219
\newSection Compare the uses of the total object and the partial object in the following examples. Notice that all three conditions for the total object must be satisfied for it to be used.

\begin{center}
Objects in Commands
\end{center}

	\twoColumnsTable
	Total Object 																& Partial Object \\
	& \\
	Võta \m{raamat}! \small{\nom \sing}				& Ära võta \m{raamatut}! \small{\parti \sing} \\
	`Take the \n{book}!'												& `Don't take the \n{book}!' \small{(negative)} \\
	& \\
	Loe \m{raamat} läbi! \small{\nom \sing}		& Loe \m{raamatut}! \small{\parti \sing} \\
	`Read the \n{book} (all the way) through!'	& `Read the \n{book}!' \small{(not necessarily to the end)} \\
	& \\
	Joo see \m{vesi} ära! \small{\nom \sing}	& Joo \m{vett}! \small{\parti \sing} \\
	`Drink (all) this \n{water} up!'						& `Drink (some) \n{water}!' \small{(indefinite quantity)}
	\tableEnd

% Section 220
\newSection

\begin{center}
Objects in Declarations
\end{center}

	\twoColumnsTable
	Total Object 																				& Partial Object \\
	& \\
	Ma võtsin \m{raamatu}. \small{\gen \sing}					& Ma ei võtnud \m{raamatut}. \small{\parti \sing} \\
	`I took the \n{book}.'															& `I did not take the \n{book}.' \small{(negative)} \\
	& \\
	Ma lugesin \m{raamatu} läbi. \small{\gen \sing}		& Ma lugesin \m{raamatut}.' \small{\parti \sing} \\
	`I read the \n{book} (all the way) through.'				& `I was reading the \n{book}.' \small{(unfinished)} \\
	& \\
	Ma sõin \m{leiva} ära. \small{\gen \sing}					& Ma sõin \m{leiba}. \small{\parti \sing} \\
	`I ate up (all) the \n{bread}.'											& `I ate (some) \n{bread}.' \small{(indefinite quantity)}
	\tableEnd

% Section 221
\newSection If there is a total object in a clause with a present/future tense verb, the verb is translated in the future tense. If it were translated in the present tense, the action would on-going or incomplete, which would require the partial object instead.

	\twoColumnsTable
	Total Object 																& Partial Object \\
	& \\
	Ma ostan \m{raamatu}. \small{\gen \sing}  & Ma ostan \m{raamatut}. \small{\parti \sing} \\
	`I will buy the \n{book}.' 									& `I am (engaged in) buying the \n{book}.'
	\tableEnd

% Section 222
\newSection Note! The total object is usually employed when the verb in an affirmative sentence is joined with a word implying the completion of the action. Such accompanying words include: \m{ära} `away', \m{läbi} `through', \m{üles} `up', \m{maha} `down', \m{valmis} `finished'

	\twoColumnsTable
	Total Object 																				& Partial Object \\
	& \\
	Ma lugesin \m{raamatu} läbi. \small{\gen \sing}		& Ma lugesin \m{raamatut}.\small{\parti \sing} \\
	`I read the \n{book} (all the way) through.'	`			& `I was reading the \n{book}.' \\
	& \\
	Ta õppis \m{eesti keele} ära. \small{\gen \sing}	& Ta õppis eesti keelt \small{\parti \sing} \\
	`He learned (to speak) \n{Estonian} (fluently).'		& `He was studying \n{Estonian}.' \\
	\tableEnd

% Section 223
\newSection Note! The partial object (in the partitive case) is always used with verbs expressing feelings, uses of the senses, or continual actions with uncertain outcomes. Such verbs include the following:

	\threeColumnsTable
	Feelings 												& Senses														& Unresolved Actions \\
	& \\
	\m{armastama} `to love' 				& \m{kuulama} `to listen' 					& \m{aitama} `to help' \\
	\m{austama} `to honor' 					& \m{kuulma} `to hear' 							& \m{juhtima} `to direct' \\
	\m{imetlema} `to admire' 				& \m{maitsma} `to taste' 						& \m{jätkama} `to continue' \\
	\m{kartma} `to fear' 						& \m{mäletama} `to remember' 				& \m{nõudma} `to demand' \\
	\m{kiitma} `to praise' 					& \m{nautima} `to enjoy' 						& \m{ootama} `to await' \\
	\m{põlgama} `to despise' 				& \m{nuusutama} `to smell' 					& \m{otsima} `to seek' \\
	\m{tundma} `to feel, know' 			& \m{nägema} `to see' 							& \m{segama} `to disturb' \\
	\m{usaldama} `to have faith in' & \m{puudutama} `to touch' 					& \m{takistama} `to hinder' \\
	\m{vihkama} `to hate'						& \m{vaatama} `to watch'						& \m{uurima} `to investigate'
	\tableEnd

Other verbs that belong in this group, especially under the third column, are \m{alustama} `to begin', \m{arvestama} `to consider', \m{kaitsma} `to defend', \m{karistama} `to punish', \m{kohtama} `to meet', \m{kõnelema} `to speak', \m{mõistma} `to understand', \m{omama} `tp own', \m{oskama} `to know how to do something', \m{rääkima} `to talk', \m{soovima} `to wish for', \m{tervitama} `to greet', \m{tänama} `to thank', \m{vajama} `to need', \m{õpetama} `to teach', \m{õppima} `to study', \m{ähvardama} `to threaten', \m{ärritama} `to irritate'.

% Section 224
\newSection With some of the above words, the total object may be used, if the verb has an accompanying word indicating that the action leads to a conclusion.

	\twoFixedColumnsTable
	Total Object 																													& Partial Object \\
	& \\
	Ma tundsin oma \m{sõbra} häälest ära. \small{\gen} 										& Ma tundsin oma \m{sõpra} hästi. \small{\parti} \\
	`I recognized my \n{friend} by his voice.' \small{(completed action)}	& `I knew my \n{friend} well.' \small{(on-going action)} \\
	Me ootasime rongi \m{tuleku} ära. \small{\gen}												& Me ootasime rongi \m{tulekul}. \small{\parti} \\
	`We waited until the coming of the train.' \small{(completed action)}	& `We awaited the coming of the train' \small{(unresolved action)}
	\tableEnd

% Section 225
\newSection Some verbal phrases always involve the partial object (in the punitive case), oven if the action is completed. Such special phrases include:

	\twoColumnsTable
	\m{abi} andma `to give help'								& \m{aset} leidma `to lake place' \\
	\m{´aega} kaotama `to lose time'						& \m{elu} nautima `to enjoy life' \\
	\m{´aega} raiskama `to waste time'					& \m{habet} ajama `to shave (one's beard)' \\
	\m{juttu} ajama `to make conversation' 			& \m{pead} raputama `to shake one's head' \\
	\m{´kihla} vedama `to make a bet' 					& \m{´rolli} mängima `to play a role' \\
	\m{lund} sadama `to (precipitate) snow' 		& \m{sõna} kuulma `to obey (a command)' \\
	\m{muret} tundma `to (feel) worry' 					& \m{sõna} pidama `to keep one's word' \\
	\m{´märku} andma `to give a sign' 					& \m{sõna} võtma `to have the floor (to speak)' \\
	\m{´nalja} tegema `to (make a) joke' 				& \m{und} nägema `to (see a) dream' \\
	\m{osa} mängima `to play a role' 						& \m{vabandust} paluma `to ask forgiveness' \\
	\m{osa} võtma `to take part'								& \m{´vihma} sadama `to (precipitate) rain'
	\tableEnd

The partial object (in the partitive case) is always used with the verb \m{mängima} `to play' when a musical instrument or game is involved, even if the action is finite or completed:

	\twoColumnsTable
	\m{´bridži} mängima `to play bridge' 		& \m{´palli} mängima `to play ball' \\
	\m{´kaarte} mängima `to play cards' 		& \m{´peitust} mängima `to play hide-and-seek' \\
	\m{klaverit} mängima `to play piano' 		& \m{tennist} mängima `to play tennis' \\
	\m{malet} mängima `to play chess' 			& \m{viiulit} mängima `to play violin'
	\tableEnd

\Title{Personal Pronouns as Direct Objects}

% Section 226
\newSection The third-person pronouns \m{tema} `he, she, it' and \m{nemad} `they' follow the same rules as nouns. That is, as a total object, the singular form \m{tema} is in the nominative case in an (affirmative) imperative sentence and in the genitive case in an (affirmative) declarative sentence; the plural form \m{nemad} appears in the nominative case as a total object. As a partial object, both the singular \m{tema} and the plural \m{nemad} appear in the partitive case (\m{teda}, \m{neid}).

	\twoFixedColumnsTable
	Total Object 																															& Partial Object \\
	& \\
	Võta \m{tema}/\m{nemad} [\m{ta}/\m{nad}] kaasa! \nom											& Ära võta \m{teda}/\m{neid} kaasa! \parti \\
	`(You \sing) Take \n{him}/\n{them} along!' \small{(affirmative command)}	& `Do not take \n{him}/\n{them} along.' \small{(negative command)} \\
	& \\
	Ma võtsin \m{tema}/\m{nemad} [\m{ta}/\m{nad}] kaasa. \gen \nom 						& Ma ei võtnud \m{teda}/\m{neid} kaasa. \parti \\
	`I took \n{him}/\n{them} along.' \small{(affirmative declaration)}				& `I did not take \n{him}/\n{them} along.' \small{(negative declaration)} \\
	\tableEnd

% Section 227
\newSection The first- and second-person pronouns behave differently than other nouns. They generally appear in the partitive case, whether they are total or partial objects. Compare:

	\twoFixedColumnsTable
	Jäta \m{mind} rahule! \parti \sing						& `(You \sing) Leave \n{me} alone!' \\
	Jäta \m{poiss} rahule! \nom \sing							& `(You \sing) Leave the \n{boy} alone!' \\
	Isa võttis \m{meid} ´linna kaasa. \parti \pl	& `Dad took \n{us} along to town.' \\
	Isa võttis \m{poisi} ´linna kaasa. \gen \sing	& `Dad took the \n{boy} along to town.'
	\tableEnd

Note: In declarative sentences, it is permitted to employ the genitive form of \m{mina} T and \m{sina} `you \sing', instead of the partitive \m{mind} and \m{sind}. But the plural forms of \m{meie} `we' and \m{teie} `you \pl' must always be in the partitive when serving as direct objects, even when they are total objects.

	\oneColumnTable
	Isa võttis \m{mind}/\m{minu} [\m{mu}] ´linna kaasa. \parti \sing or \gen \sing \\
	Isa võttis \m{meid} ´linna kaasa. (part. pi. only) \\
	\\
	Isa võttis \m{sind}/\m{sinu} [\m{su}] ´linna kaasa, \parti \sing or \gen \sing  \\
	Isa võttis \m{teid} ´linna kaasa. \parti \pl (only) \\
	\\
	Isa võttis \m{tema} [\m{ta}] ´linna kaasa. \gen \sing (only) \\
	Isa võttis \m{nemad} [\m{nad}] ´linna kaasa. \nom \pl (only) \\
	\tableEnd

\Vocabulary % ==================================================================

\Exercises % ===================================================================

\Expressions % =================================================================

\AnswersToExercises % ==========================================================