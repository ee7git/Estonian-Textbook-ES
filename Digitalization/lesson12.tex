
\newLesson % Lesson 12

\Grammar % =====================================================================

\Title{Numbers \cont}

Estonian numbers above 10 are formed very regularly. All you need to know are the basic numbers 1-10 (see Lesson 5). The case forms of numbers will be discussed in Lesson 33. \\

% Section 61
\newSection The numbers from 11 to 19 are combinations of the basic numbers 1-9 and the word \m{teist} or \m{teistkümmend}, which means `of the second ten'. (This suffix is similar to the English `teen', except that it applies to 11 and 12 as well as 13-19.) The longer form \m{teistkümmend} is rare in the modern language.

	\twoColumnsTable
	\m{11 üks + teist(kümmend) = üksteist(kümmend)} 	& \\
	\m{12 kaksteist}									& \m{16	kuusteist} \\
	\m{13 kolmteist}									& \m{17	seitseteist} \\
	\m{14 neliteist}									& \m{18	kaheksateist} \\
	\m{15 viisteist}									& \m{19	üheksateist}
	\tableEnd

% Section 62
\newSection The numbers 20, 30, etc. are constructed by adding the suffix \m{kümmend} (like the
English -ty for	`ten') to 2, 3, etc.	

	\twoColumnsTable
	\m{20 kakskümmend} & \m{60 kuuskümmend} \\
	\m{30 kolmkümmend} & \m{70 seitsekümmend} \\ 
	\m{40 nelikümmend} & \m{80 kaheksakümmend} \\ 
	\m{50 viiskümmend} & \m{90 üheksakümmend}
	\tableEnd

% Section 63
\newSection The numbers 21, 22, and so on are constructed in the same way as in English (twenty + one =	= twenty-one). In Estonian, the two figures	are written as separate words:

	\twoColumnsTable
	\m{21 kakskümmend üks}	& \m{31 kolmkümmend üks} \\
	\m{22 kakskümmend kaks}	& \m{37 kolmkümmend seitse} \\
	\m{23 kakskümmend kolm}	& \m{58 viiskümmend kaheksa} \\
	\m{24 kakskümmend neli}	& \m{76 seitsekümmend kuus} \\
	\m{ete.} 				& \m{99 üheksakümmend üheksa}
	\tableEnd

% Section 64
\newSection Higher numbers are formed as follows:

	\twoFixedColumnsTable
	\m{100 sada}							& \m{1 000 (üks) tuhat} \\
	\m{200 kakssada}						& \m{2 000 kaks tuhat} \\
	\m{300 kolmsada}						& \m{3 000 kolm tuhat} \\
	& \\
	\m{101 sada üks}						& \m{1 001 tuhat üks} \\
	\m{102 sada kaks}						& \m{1 120 tuhat ükssada kakskümmend } \\
	\m{312 kolmsada kaksteist}				& \m{10 000 kümme tuhat } \\
	\m{753 seitsesada viis-kümmend kolm}	& \m{100 000 sada tuhat} \\
	& \\
	\m{1 000 000 (üks) miljon}				& \\
	\m{2 000 000 kaks miljonit}				& \\
	\m{10 000 000 kümme miljonit}			& \\
	\m{1 000 000 000 (üks) miljard}			& \\
	\m{2 000 000 000 kaks miljardit}		& 
	\tableEnd

% Section 65
\newSection Note! The suffixes \m{-teist(kümmend)}, \m{-kümmend}, and \m{-sada} are written together with the number they apply to, as in \m{kaksteist(kümmend)}, \m{kakskümmend}, \m{kakssada}. In all other instances, the words are separated when written, as in \m{kaks tuhat}, \m{kakskümmend üks}, \m{sada viisteist}. \\

	\twoColumnsTable
	\m{147}			& \m{sada nelikümmend seitse} \\
	\m{321}			& \m{kolmsada kakskümmend üks} \\
	\m{718}			& \m{seitsesada kaheksateist} \\
	\m{1987}		& \m{tuhat üheksasada kaheksakümmend seitse} \\
	\m{7549}		& \m{seitse tuhat viissada nelikümmend üheksa}
	\tableEnd

\Title{Partitive Case after Numbers}

% Section 66
\newSection Following the number 1, a noun is in the nominative singular case: üks \m{aasta} `one year', üks \m{tund} `one hour', üks \m{minut} `one minute', üks \m{dollar} `one dollar', üks \m{kroon} `one crown' [Estonian currency]. \\

After all other numbers, the noun is in a special case form called the \n{partitive singular}. This is one of the basic cases. Like the genitive singular, it is used to construct many other cases (see Lesson 30). In English, by way of contrast, the noun takes the \n{plural} form when a quantity of two or more is involved (2 years, 3 hours, 100 minutes, 101 crowns). \\

Some examples:

	\sixColumnsTable
	(2) 		& kaks			& aastat,	& ´tundi, 	& minutit, 	& ´krooni \\
	(3) 		& kolm			& \(\gg\)	& \(\gg\)	& \(\gg\)	& \(\gg\) \\
	(100) 		& sada			& \(\gg\)	& \(\gg\)	& \(\gg\)	& \(\gg\) \\
	(101) 		& sada üks		& \(\gg\)	& \(\gg\)	& \(\gg\)	& \(\gg\) \\
	(1000) 		& tuhat			& \(\gg\)	& \(\gg\)	& \(\gg\)	& \(\gg\) \\
	(1/2) 		& pool			& \(\gg\)	& \(\gg\)	& \(\gg\)	& \(\gg\) \\
	(1/4) 		& veerand		& \(\gg\)	& \(\gg\)	& \(\gg\)	& \(\gg\) \\
	(3/4) 		& kolmveerand	& \(\gg\)	& \(\gg\)	& \(\gg\)	& \(\gg\) \\
	(1 1/2) 	& poolteist		& \(\gg\)	& \(\gg\)	& \(\gg\)	& \(\gg\) \\
	(a pair) 	& paar			& \(\gg\)	& \(\gg\)	& \(\gg\)	& \(\gg\) \\
	(many) 		& mitu			& \(\gg\)	& \(\gg\)	& \(\gg\)	& \(\gg\) \\
	\tableEnd

Pilet maksab \m{üks dollar} `The ticket costs one dollar'. Raamat maksab \m{kümme dollarit} `The book costs ten dollars'. Ta on \m{kakskümmend aastat} vana `He/she is twenty years old'. Üks akadeemiline tund on \m{nelikümmend viis minutit} `A class hour is forty-five minutes'.

\m{Mitu dollarit [`krooni]} see maksab? `How many dollars [crowns] does it cost?' \m{Mitu aastat}? `How many years?'

\Text % ========================================================================

\Vocabulary % ==================================================================

\Exercises % ===================================================================

\Expressions % =================================================================

\AnswersToExercises % ==========================================================