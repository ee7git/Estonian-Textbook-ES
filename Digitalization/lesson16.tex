
\newLesson % Lesson 16
\label{lesson-16}

\Grammar % =====================================================================

\Title{Outer Locative Cases}

% Section 92
\newSection \label{section-92} Estonian has three outer locative cases which, like the inner locative cases (Lessons \ref{lesson-14}-\ref{lesson-15}), answer the questions where to (whither)?, where?, where from (whence)? The outer locative cases indicate location or movement in relation to the surface or the environment of something. \\

The outer locative cases are constructed by adding the endings \m{-le}, \m{-l}, \m{-It } to the genitive form.

	\twoColumnsTable
	\m{tänava + le} 	& = (out) onto the street \\
	\m{tänava + l} 	& = on the street \\
	\m{tänava + lt} 	& = (off) from the street \\
	& \\
	\m{töö + le} 		& = to work \\
	\m{töö + l} 		& = at work \\
	\m{töö + lt} 		& = from work 
	\tableEnd

\Title{The Allative Case}

% Section 93
\newSection \label{section-93} The allative (called \m{alaleütlev} or onto-saying case in Estonian) indicates movement toward something. It has the ending \m{-le}, which is added to the genitive case of a word.

	\fourColumnsTable
	\n{Nominative} &		& \n{Genitive} 	& \n{Allative} \\
	laud	& `table, board'	& \m{laua} 		& \m{laua/le} \\
	katus	& `roof'		& \m{katuse}		& \m{katuse/le} \\
	pink	& `bench'		& \m{pingi}		& \m{pingi/le} \\
	Peeter	& `Peter'		& \m{Peetri}		& \m{Peetri/le}
	\tableEnd

This case has two basic meanings:

% Section 94
\newSection \label{section-94} a) The allative indicates the object onto the surface of which a movement occurs. It answers the questions \m{kuhu?} `where to (whither)?', \m{kellele?} `on(to) whom?', and \m{millele?} `on(to) what?'.

	\twoColumnsTable
	Pane raamat \m{lauale}.	& `Put the book on(to) the table.' \\
	Istu \m{pingile}.		& `Sit (down) on the bench.' \\
	Ma koputan \m{uksele}.	& `I knock on the door.' 
	\tableEnd

% Section 95
\newSection \label{section-95} b) The allative indicates the person to whom one gives or says something. It answers to questions \m{kellele?} `to whom?', and \m{millele?} `to what?'. (Here the allative is comparable to the dative case in German, among other languages.)

	\twoColumnsTable
	Anna raamat \m{lapsele}.		`Give the book to the child.' \\
	Kirjuta \m{vennale} üks kiri. 		`Write a letter to Brother.' \\
	Olen \m{sinule} väga tänulik. 	`I am very grateful to you.' \\
	Ütle \m{sõbrale}, et ta siia tuleks. 	`Tell your friend to come here.'
	\tableEnd

\Title{The Adessive Case}

% Section 96
\newSection \label{section-96}The adessive (\m{alalütlev} or on-saying) case indicates location on top of something. It has the ending \m{-l}, which is added to the genitive form of a word.

	\threeColumnsTable
	\n{Nominative}	& \n{Genitive}	& \n{Adessive} \\
	sein `wall'		& \m{seina}		& \m{seina/l} \\
	talv `winter'		& \m{talve}		& \m{talve/l} \\
	tool `chair'		& \m{tooli}		& \m{tooli/l} \\
	põld `field'		& \m{põllu}		& \m{põllu/l} \\
	Jaan `John'		& \m{Jaani}		& \m{Jaani/l} \\
	\tableEnd

% Section 97
\newSection \label{section-97} In its basic sense, the adessive indicates the object on the surface of which something is found. It answers the questions \m{kus?} `where (at)?', \m{kellel?} `on (top of) whom?', and \m{millel?} `on (top of) what?'.

	\twoColumnsTable
	Põllumees töötab \m {põllul}.	& `The farmer works in [on] the field.' \\
	Ma istun \m{toolil}.			& `I am sitting on the chair.' \\
	Pilt ripub \m{seinal}.			& `The picture is hanging on the wall.'
	\tableEnd

% Section 98
\newSection \label{section-98} The adessive is often used in expressions of time. In many cases, the English equivalent would involve the preposition `in', rather than `on': \m{talvel} `in winter',  \m{hommikul} `in the morning', \m{õigel ajal} `at the right time', \m{tuleval aastal} `next year'. (See the discussion of this topic in Lesson \ref{lesson-13}). \\

Another special use of the adessive (with the verb \m{olema} `to be') is to indicate possession or ownership of something, where English uses the verb `to have': \m{minul on} `I have'. This is discussed in Lesson \ref{lesson-17}.

\Title{The Ablative Case}

% Section 99
\newSection \label{section-99} The ablative (\m{alaltütlev} or off-saying) case indicates movement off or away from something. It ends in \m{-It}, which is added to the genitive.

	\fourColumnsTable
	\n{Nominative}	&			& \n{Genitive}	& \n{Ablative} \\
	tänav  			& `street'		& \m{tänava}		& \m{tänava/lt} \\
	laev			& `ship'		& \m{laeva}		& \m{laeva/lt} \\
	õde			& `sister'		& \m{õe}		& \m{õe/lt} \\
	Maret 			& woman's name 	& \m{Mareti}		& \m{Mareti/lt}
	\tableEnd	

The ablative is basically used in two senses:

% Section 100
\newSection \label{section-100} a) It indicates the object from the surface of which a movement occurs. It answers the questions \m{kust?} `where from? whence?', \m{kellelt?} `off whom? away from whom?', and \m{millelt?} `off what? away from what?'.

	\twoColumnsTable
	Poiss tuleb \m{tänavalt} tuppa.	& `The boy comes indoors from the street.' \\
	Võta \m{laualt} pliiats.		& `Take the pencil off (from) the table.' \\
	Rotid lahkuvad \m{laevalt}.		& `The rats leave [go off] the ship.'
	\tableEnd

% Section 101
\newSection \label{section-101} b) The ablative indicates the person from whom one receives, takes, or demands something. It answers the question \m{kellelt?} `from whom? of whom?'.

	\twoColumnsTable
	See kiri on \m{Maretilt}.		& `This letter is from Maret.' \\
	Ma laenan \m{sõbralt} raha.		& `I borrow money from a friend.' \\
	Küsi \m{vennalt}, millal ta tuleb.  	& `Ask (of) Brother when he's coming.'
	\tableEnd

% Section 102
\newSection \label{section-102} The adjective that modifies a noun in the allative, adessive, or ablative case must be in the same case form as the noun. \\

Ära istu \m{märjale pingile}! `Don't (you \sing) sit down on the wet bench!' Me elame \m{rahutul ajal} `We live in an uneasy time'. Onu saabus \m{pikalt reisilt} `Uncle arrived from a long trip'.

\Title{Personal Pronouns}

For personal pronouns, the allative, adessive, and ablative cases are basically made by adding the normal case endings to the genitive form (See \textsection \ref{section-49} in Lesson \ref{lesson-8}). Some simplifications and modified spellings exist in the plural and short forms.

	\twoColumnsTable
	\n{Allative}			& \n{Adessive}		& \n{Ablative} \\
	\m{minule -- mulle} 		& \m{minul -- mul}		& \m{minult -- mult} \\
	\m{sinule -- sulle}		& \m{sinul -- sul} 		& \m{sinult -- sult } \\
	\m{temale -- talle}	  	& \m{temal -- tal}	 	& \m{temalt -- talt} \\
	& & \\
	\m{meile}			& \m{meil}			& \m{meilt} \\
	\m{teile}			& \m{teil}			& \m{teilt} \\
	\m{nendele -- neile}	& \m{nendel -- neil}		& \m{nendelt -- neilt} 
	\tableEnd

Ulata \m{mulle} see raamat! `Give me that book!' Ta palub \m{meilt} abi `He is asking us for help.' Tule homme \m{meile} `Come to our place [\lit to us] tomorrow'. Mine \m{neile} külla `Go visit them [on a visit to them].'

\Text % ========================================================================

\Vocabulary % ==================================================================

\Exercises % ===================================================================

\Expressions % =================================================================

\AnswersToExercises % ==========================================================