\newLesson % Lesson 30

\Grammar % =====================================================================

\Title{Plural Declension}

% Section 241
\newSection There are 14 case forms for nouns and adjectives in the plural, just as there are in the singular. Plural case forms are derived by similar principles. That is, there are only three basic or ``real'' cases--nominative, genitive, partitive--and the rest are obtained by adding certain endings to one of these. These endings are the same in the singular and plural, and they have the same meaning or function, as equivalents to prepositions in English. For example, the suffix \m{-ga} means `with' and the suffix \m{-ta} means `without', in both the singular and plural. What distinguishes singular and plural case forms is thus not the endings but the stems to which the endings are attached.

% Section 242
\newSection The nominative plural (Lesson 11) is formed by adding \m{-d} to the genitive singular. For example, \nom \sing \m{vend} `brother' > \gen \sing \m{venna} `brother's' > \nom \pl \m{venna/d} `brothers'.

% Section 243
\newSection The genitive plural plays the same role in forming other cases as the genitive singular does. (See review of the latter in Lesson 26.) It is a basic form, from which most other cases are derived. All cases in the plural, with the exception of the nominative and partitive, are constructed simply by adding the usual endings to the genitive plural form.

% Section 244
\newSection The following example shows how the noun \m{poeg} `son' is declined in the singular and plural. (Only the partitive plural remains to be discussed. See Lesson 31.)

	\fourColumnsTable
	\n{Case}					& \n{Meaning} 								& \n{Singular} 		& \n{Plural} \\
	& & & \\
	\n{Nominative}		& \m{kes?} who? 							& \m{poeg} `son'  & \m{poja/d} `sons' \\
	\n{Genitive}			& \m{kelle?} whose? 					& \m{poja} 				& \m{poega/de} \\
	\n{Partitive}			& \m{keda?} whom? 						& \m{poega} 			& \m{poegi} \\
	& & & \\
	\n{Illative}			& \m{kellesse?} into whom? 		& \m{poja/sse} 		& \m{poega/de/sse} \\
	\n{Inessive}			& \m{kelles?} in(side) whom? 	& \m{poja/s} 			& \m{poega/de/s} \\
	\n{Elative}				& \m{kellest?} out of whom? 	& \m{poja/st} 		& \m{poega/de/st} \\
	& & & \\
	\n{Allative}			& \m{kellele?} (on)to whom? 	& \m{poja/le} 		& \m{poega/de/le} \\
	\n{Adessive}			& \m{kellel?} at/on whom? 		& \m{poja/1} 			& \m{poega/de/1} \\
	\n{Ablative}			& \m{kellelt?} off/from whom? & \m{poja/lt} 		& \m{poega/de/lt} \\ %ISSUE pojal/t > poja/lt 
	& & & \\
	\n{Translative}		& \m{kelleks?} becoming whom? & \m{poja/ks} 		& \m{poega/de/ks} \\
	\n{Terminative}		& \m{kelleni?} until whom? 		& \m{poja/ni} 		& \m{poega/de/ni} \\
	\n{Essive}				& \m{kellena?} as who? 				& \m{poja/na} 		& \m{poega/de/na} \\
	& & & \\
	\n{Comitative}		& \m{kellega?} with whom? 		& \m{poja/ga} 		& \m{poega/de/ga} \\
	\n{Abessive}			& \m{kelleta?} without whom? 	& \m{poja/ta} 		& \m{poega/de/ta} \\
	\tableEnd

% Section 245
\newSection From the table above, it is apparent that declension of the noun involves two different stems: \m{poja-} and \m{poega-}. One of these, called the \n{genitive stem} (\m{poja-}) because it is the same as the genitive singular form, is the basis of declension in the singular. The other, called the \n{partitive stem} (\m{poega-}) because it is the same as the partitive singular form, forms the basis of declension in the plural.

% Section 246
\newSection The genitive plural always has the ending \m{-de} or \m{-te} (Lesson 29). The element \m{-de/-te} at the end of a word or just before the preposition-like ending is thus an indicator that the word is in the plural. \\

A word that is declined in the plural generally consists of the following parts, which are easy to distinguish. (Exceptional patterns exist for the words presented in \textsection 232-235.)

	\begin{center}
	\m{PARTITIVE STEM + PLURAL INDICATOR -de/-te + CASE ENDING}
	\end{center}

For example:

	\fourColumnsTable
	 	\m{poega} 		& \m{+ de} & \m{+ ga} &	`with sons' \\
		\m{õpilas(t)} & \m{+ te} & \m{+ ta} &	`without students'
	\tableEnd

% Section 247
\newSection Here are some more examples of how nouns are declined in both the singular and plural. The words are \m{mees} `man', \m{sõber} `friend', and \m{eestlane} `Estonian'. The partitive plural is shown in parentheses, since it has not yet been discussed.

	\threeColumnsTable
	\n{Case}				& \n{Singular Forms}								& \n{Plural Forms} \\
	& & \\
	\n{nominative}	& mees, sõber, eestlane 						& mehed, sõbrad, eestlased \\
	\n{genitive}		& \m{mehe, sõbra, eestlase}					& \m{meeste, sõprade, eestlaste} \\
	\n{partitive}		& \m{meest, sõpra, eestlast}				& (mehi, sõpru, eestlasi) \\
	& & \\
	\n{illative}		& mehesse, sõbrasse, eestla(se)sse 	& meestesse, sõpradesse, eestlastesse  \\
	\n{inessive}		& mehes, sõbras, eestlases 					& meestes, sõprades, eestlastes  \\
	\n{elative}			& mehest, sõbrast, eestlasest				& meestest, sõpradest, eestlastest \\
	& & \\
	\n{allative}		& mehele, sõbrale, eestlasele 			& meestele, sõpradele, eestlastele  \\ %ISSUE eestastele > eestlastele
	\n{adessive}		& mehel, sõbral, eestlasel 					& meestel, sõpradel, eestlastel  \\ %ISSUE eestastel > eestlastel
	\n{ablative}		& mehelt, sõbralt, eestlaselt				& meestelt, sõpradelt, eestlastelt \\ %ISSUE eestastelt > eestlastelt
	& & \\
	\n{translative}	& meheks, sõbraks, eestlaseks 			& meesteks, sõpradeks, eestlasteks  \\
	\n{terminative}	& meheni, sõbrani, eestlaseni 			& meesteni, sõpradeni, eestlasteni  \\
	\n{essive}			& mehena, sõbrana, eestlasena				& meestena, sõpradena, eestlastena \\
	& & \\
	\n{comitative}	& mehega, sõbraga, eestlasega 			& meestega, sõpradega, eestlastega  \\
	\n{abessive}		& meheta, sõbrata, eestlaseta				& meesteta, sõpradeta, eestlasteta \\
	\tableEnd

Adjectives in the plural are declined in the same way as in the singular. They generally take the same forms as the nouns they modify. For the last four case forms in the list above, the adjective remains in the genitive form, and only the noun takes the \m{-ni}, \m{-na}, \m{-ga}, or \m{-ta} ending. (Lesson 33)

% Section 248
\newSection Here are some further examples of how nouns may be declined. The words are \m{jalg} `foot, leg’, \m{käsi} `hand, arm', and \m{tuba} `room'.

	\threeColumnsTable
	\n{Case}							& \n{Singular Forms}			& \n{Plural Forms} \\
	& & \\
	\n{Nominative}				& jalg, käsi, tuba 				& jalad, käed, toad  \\
	\n{Genitive}					& \m{jala, käe, toa}			& \m{jalgade, käte, tubade} \\
	\n{Partitive}					&	\m{jalga, kätt, tuba}		& (jalgu, käsi, tube) \\
	& & \\
	\n{Illative}					& jalasse, käesse, toasse & jalgadesse, kätesse, tubadesse \\
	\n{(Short Ill.)}			& (jalga, kätte, tuppa) 	& \\
	\n{Inessive}					& jalas, käes, toas 			& jalgades, kätes, tubades  \\
	\n{Elative}						& jalast, käest, toast		& jalgadest, kätest, tubadest \\
	& & \\
	\n{Allative}					& jalale, käele, toale 		& jalgadele, kätele, tubadele  \\
	\n{Adessive}					& jalal, käel, toal 			& jalgadel, kätel, tubadel  \\
	\n{Ablative}					&	jalalt, käelt, toalt		& jalgadelt, kätelt, tubadelt \\
	& & \\
	\n{Translative}				& jalaks, käeks, toaks 		& jalgadeks, käteks, tubadeks  \\
	\n{Terminative}				& jalani, käeni, toani 		& jalgadeni, käteni, tubadeni  \\
	\n{Essive}						& jalana, käena, toana		& jalgadena, kätena, tubadena \\
	& & \\
	\n{Comitative}				& jalaga, käega, toaga 		& jalgadega, kätega, tubadega  \\
	\n{Abessive}					& jalata, käeta, toata		& jalgadeta, käteta, tubadeta
	\tableEnd

% Section 249
\newSection When two or more plural nouns are mentioned in a series, the case endings for the comitative and abessive are usually just added to the last noun, while the others remain in the genitive without the ending. This is the same pattern that occurs in the singular (Lesson 20).

	\oneColumnTable
	Ma sõdisin vastu \m{käte(ga)} ja \m{jalgadega}, \m{küünte(ga)} ja \m{hammastega}. \\
	`I fought back with hands and (with) feet, with nails and (with) teeth.'
	\tableEnd

% Section 250
\newSection In Estonian as in English, there are some words that always appear in the plural form. These include \m{püksid} `pants', \m{prillid} `eyeglasses', \m{käärid} `scissors', \m{tangid} `pliers', \m{riided} `clothes, garments'. (The singular form \m{riie} means `cloth, textile'). \\

Some nouns that occur in the plural in Estonian are translated into the singular in English. These include \m{juuksed} `hair', \m{teadmised} `knowledge', \m{ilmad} `weather', \m{pulmad} `wedding (rites)', \m{matused} `funeral (rites)'. \\

\m{Juuksed} tõusevad püsti `The \n{hair} stands on end'. \m{Teadmised} on võim `\n{Knowledge} is power'. Suvel on \m{ilmad} ilusad `In summer the \n{weather} is nice'. (But: Täna on ilus \m{ilm} `Today the \n{weather} is nice'.) \m{Pulmad} toimuvad laupäeval `The \n{wedding} will take place on Saturday'. \m{Matused} on reedel `The \n{funeral} will be on Friday.' \\ % ISSUE funderal > funeral

Note! The word \m{raha}, when it means `money' (rather than bills), appears in the singular as it would in English: Palju \m{raha} `Lots of money'.

% Section 251
\newSection Notice the following instances where a noun, which stands for several objects belonging to different owners, appears in the singular instead of the plural form that would be used in English: \\

Lapsed on õues ilma \m{mütsita} `The children are outside \n{without} their \n{hats}'. Nad tulid sisse raamatud \m{käes} `The came in with books \n{in} their \n{hands}'. (But: \m{kätel} kandma `to carry with the \n{arms}') Me jooksime \m{elu} eest `We ran for our \n{lives}'.

\Vocabulary % ==================================================================

\Exercises % ===================================================================

\Expressions % =================================================================

\AnswersToExercises % ==========================================================