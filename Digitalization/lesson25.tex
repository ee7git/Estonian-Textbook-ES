\newLesson % Lesson 25
\label{lesson-25}

\Grammar % =====================================================================

\Title{The Translative Case}

% Section 174
\newSection \label{section-174} The translative case (called \m{saav} or `becoming' in Estonian) as such indicates what someone or something is turning into. It answers the questions \m{kelleks?} `becoming whom?' and \m{milleks?} `becoming what?'. In other words, the translative case expresses a change in identity or condition. \\

The translative case is formed by adding the suffix \m{-ks} to the genitive form.

	\threeColumnsTable
	\n{Nominative}			& \n{Genitive} 		& \n{Translative} \\
	õpetaja `teacher'		&	\m{õpetaja}			& \m{õpetaja/ks} \\
	mees `man, husband'	&	\m{mehe}				& \m{mehe/ks} \\
	naine `woman, wife'	& \m{naise}				& \m{naise/ks} \\
	ilus `beautiful'		&	\m{ilusa}				& \m{ilusa/ks} \\
	suur `big'					&	\m{suure}				& \m{suure/ks} \\
	tugev `strong'			&	\m{tugeva}			& \m{tugeva/ks} \\
	vana `old'					&	\m{vana}				& \m{vana/ks} 
	\tableEnd

	\twoFixedColumnsTable
	Mu vend tahab saada \m{õpetajaks}. 					& `My brother wants to become a \n{teacher}.' \\
	See noor neiu tahab saada \m{filmitäheks}.	& `That young maiden wants to become a \n{film star}.' \\
	Igaüks saab \m{õndsaks} omal viisil. 				& `Everyone becomes \n{happy} in his own way.' \\
	Ilm on läinud \m{ilusaks}.									& `The weather has turned \n{beautiful}.' \\
	Sa oled jäänud \m{vanaks}.									& `You \sing have gotten \n{old}.' \\
	Laps jäi eile \m{haigeks}.									& `The child got \n{sick} yesterday.' \\
	Ta sai nelikümmend aastat \m{vanaks}.				& `He turned forty (years \n{old}).' 
	\tableEnd

% Section 175
\newSection \label{section-175} In addition, the translative case has the following usages. \\

a) in conjunction with the verbs \m{tegema} `to make', \m{olema} `to be', and \m{lugema} or \m{pidama} in the sense of `to consider (as)', in expressions of this sort:

	\twoFixedColumnsTable
	See teeb mind \m{õnnelikuks}. 		& `That makes me (become) \n{happy}.' \\
	\m{Kelleks} te mind peate?				& `\n{Who} do you think I am? [What do you take me \n{for}?].' \\
	Ta peab [loeb] mind \m{rumalaks}.	& `She considers me (to be) \n{stupid}.' \\
	Sa oled meile \m{eeskujuks}. 			& `You \sing are a model [\n{example}] for us.' \\
	Tütar oli emale \m{abiks}.				& `The girl was a \n{help} to her mother.'
	\tableEnd

% Section 176
\newSection \label{section-176} b) to indicate the purpose or goal of the action of a verb:

	\twoFixedColumnsTable
	Meil on \m{sõiduks} raha vaja. 								& `We need money \n{for the trip}.' \\
	\m{Vastuseks} teie kirjale teatan \dots 			& `\n{In response} to your \pl letter I inform you \dots' \\
	Poeg sai isalt \m{kingituseks} ilusa raamatu.	& `The son got a beautiful book \n{as a present} from his father.' 
	\tableEnd

% Section 177
\newSection \label{section-177} c) to indicate the time during which or by which something occurs. Here, the translative case answers the questions \m{kui kauaks?} `for how long' and \m{mis ajaks?} `by when?'.

	\twoFixedColumnsTable
	\m{Kui kauaks} sa siia jääd?											& `\n{How long} will you \sing stay here?' \\
	Ma jään \m{üheks nädalaks} [\m{kaheks tunniks}]. 	& `I will stay for \n{one week} [\n{two hours}].' \\
	Maja saab valmis \m{kevadeks}.	 									& `The house will be finished \n{by spring}.' \\
	Isa lubas tulla \m{õhtuks} koju.									& `Dad promised to be home \n{by evening}.'
	\tableEnd

% Section 178
\newSection \label{section-178} d) to indicate the order in which something occurs.

	\twoFixedColumnsTable
	Ta tuli (jooksus) \m{esimeseks}.					& `She came in \n{first} (in the race).' \\
	Ma jäin \m{viimaseks}.										& `I came in \n{last} [I was in \n{last} place].' \\
	\m{Esiteks}, \m{teiseks}, \m{kolmandaks}. & `\n{First of all}, \n{second}, \n{third}.'
	\tableEnd

% Section 179
\newSection \label{section-179} Note that the translative is used in many verbal phrases:

	\oneColumnTable
	\m{andeks paluma}, andeks paluda, palun andeks `to ask forgiveness' \\
	\m{andeks andma}, andeks anda, annan andeks `to forgive' \\
	\m{hiljaks jääma}, hiljaks jääda, jään hiljaks `to be(come) late' \\
	\m{kindlaks tegema}, kindlaks teha, teen kindlaks `to make sure' \\
	\m{pahaks panema}, pahaks panna, panen pahaks `to take offense' \\
	\m{paremaks pidama}, paremaks pidada, pean paremaks `to prefer [consider as better]' \\
	\m{puhtaks pesema}, puhtaks pesta, pesen puhtaks `to wash clean' \\
	\m{mustaks tegema}, mustaks teha, teen mustaks `to soil [get dirty]' \\
	\m{valgeks värvima}, valgeks värvida, värvin valgeks `to whitewash' \\
	\m{heaks kiitma}, heaks kiita, kiidan heaks `to approve [praise as good]'
	\tableEnd

% Section 180
\newSection \label{section-180} The adjective modifying a noun in the translative case must agree with the noun. That is, the adjective must also be in the translative case.

	\twoFixedColumnsTable
	Poeg on kasvanud \m{suureks meheks}. 			& `The son has grown up to be a \n{big man}.' \\
	Ma loen sind \m{õnnelikuks inimeseks}. 		& `I count [consider] you a \n{happy person}.' \\
	Me tellisime ajalehe \m{terveks aastaks}. & `We subscribed to the paper \n{for a whole year}.'
	\tableEnd

% Section 181
\newSection \label{section-181} Note! An adjective in the translative case which is in the predicate of a sentence usually is in the singular case, even if the subject of the sentence is in the plural.

	\twoFixedColumnsTable
	Laps on kasvanud \m{suureks}. 	& `The child has grown up [\n{big}].' \\
	Lapsed on kasvanud \m{suureks}. & `The children have grown up [\n{big}].' \\
	Päev läheb \m{lühemaks}. 				& `The day is getting \n{shorter}.' \\
	Päevad lähevad \m{lühemaks}. 		& `The days are getting \n{shorter}.'
	\tableEnd

This differs from the pattern when the predicate complement is in the nominative case (Lesson 11):

	\twoFixedColumnsTable
	Lapsed on \m{suured}.			& `The children are \n{big}.' \\
	Päevad on \m{lühikesed}.	& `The days are \n{short}.'
	\tableEnd

% Section 182
\newSection \label{section-182} Personal pronouns are declined regularly in the translative case. This is, the \m{-ks} ending is added to the genitive form:

	\twoColumnsTable
	\m{minu/ks}	& \m{meie/ks} \\
	\m{sinu/ks}	& \m{teie/ks} \\
	\m{tema/ks}	& \m{nende/ks} 
	\tableEnd

There are no short forms (mu/ks, su/ks, etc.) in the translative case.

\Title{The Emphatic Particle -ki/-gi}

% Section 183
\newSection \label{section-183} The emphatic particle \m{-ki} or \m{-gi} can be added to almost any word to which you want to give special emphasis or draw special attention. Which of these two forms is used depends on the letter at the end of the word to which the particle is added. \\

Consonants which are voiceless in Estonian (b, d, f, g, h, k, p, s, š, t) are followed by \m{-ki}. For instance: \m{park} `park' \m{+ ki = parkki} `even the park', \m{poeg} `son' \m{+ ki = poegki} `even the son', \m{lind} `bird' \m{+ ki = lindki} `even the bird.' \\

Consonants which are voiced in Estonian (l, m, n, r, z, ž, v) and all vowels (a, e, i, o, u, õ, ä, ö, ü) are followed by \m{-gi}. For example: \m{maja} `house' \m{+ gi = majagi} `even the house', \m{linn} `town' \m{+ gi = linngi} `even the town', \m{orav} `squirrel' \m{+ gi = oravgi} `even the squirrel'. \\

% Section 184
\newSection \label{section-184} The particle \m{-ki} or \m{-gi} can often be translated into English as `even' or `indeed'. In a negative sentence, it can mean `(not) even' or `(not/none) at all' Sometimes, however, it is impossible to give an exact counterpart in English. \\

The emphatic particle is used quite often in Estonian. You should notice the many different ways it can be employed, in order to learn its usage. Here are some examples:

	\twoColumnsTable
	\m{Minagi} olen seal olnud. 								& `\n{Even} I have been there.' \\
	Isa \m{ongi} juba kodus. 										& `Dad \p{\n{is}} (\n{indeed}) home already.' \\
	Ta on \m{merelgi} olnud. 										& `She has been \n{at sea even}.' \\
	Ma \m{ei teagi} veel, kas ma seda tahan.  	& `I \n{don't even know} yet whether I want it.' \\
	Ma \m{ei mõtlegi} seda teha. 								& `I \n{don't intend} to do that \n{at all}.' \\
	See \m{ei tule} kõne \m{allagi}. 						& `That \n{won't even come under} consideration.' \\
	Mul \m{pole mitte sentigi}. 								& `I \n{don't have a red} [\n{single}] \n{cent}' \\
	Mul \m{pole aimugi}. 												& `I \n{have \p{no} idea (at all)}.' \\
	Mul \m{pole vähematki} aimu. 								& `I \n{don't have the \p{least} idea}.' \\
	Eesti keelt rääkida \m{polegi} nii raske.  	& `Speaking Estonian \n{isn't} so hard \n{after all}.' \\
	Ta \m{ei} ütelnud \m{sõnagi} ja läks. 			& `He did \n{not} say a (\n{single}) word and left.' 
	\tableEnd

% Section 185
\newSection \label{section-185} In some cases, the emphatic particle has become a permanent part of the word: \m{siiski} `even so', \m{iialgi} `ever', \m{kunagi} `sometime', \m{ei kunagi} `never', \m{isegi} `even', \m{keegi} `someone', \m{miski} `something', \m{ei ükski} `no one', \m{kumbki} `either (one)'. \\

When the word is declined, the emphatic particle should he considered us a separate word and placed after the case ending of the main word.

	\twoFixedColumnsTable
	\m{Keegi} tuleb. 																						& `\n{Someone} is coming.' \\
	Kerge pole \m{kellel/gi}. 																	& `It is not easy \n{for anyone}.' \\
	Ta läks \m{kellega/gi} jalutama. 														& `She went for a walk \n{with someone}.' \\
	Me ei rääkinud sinust, me rääkisime \m{kellest/ki} teisest. & `We were not talking about you, we were talking \n{about someone else}.' 
	\tableEnd

(In the spoken language, the forms \m{kellegi/le}, \m{kellegi/st}, \m{kellegi/ga} are often used, though technically incorrect.)

	\twoFixedColumnsTable
	\m{Üks/ki} ei pääse oma saatusest. 											& `\n{No one} escapes (from) their fate.' \\
	Ma ei saa \m{ühest/ki} sõnast aru.											& `I do \n{not} understand a \n{single} word (of it).' \\
	Ma ei ole \m{ühele/gi} [\m{kellele/gi}] seda lausunud. 	& `I did \n{not} say that \n{to anyone} (\n{at all}).'
	\tableEnd

\Text % ========================================================================

\Vocabulary % ==================================================================

\Exercises % ===================================================================

\Expressions % =================================================================

\AnswersToExercises % ==========================================================

