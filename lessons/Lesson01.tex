% Lesson 1

\chapter{Primera Lección} % Chapter title

\label{ch:lesson01} % For referencing the chapter elsewhere, use \autoref{ch:examples} 

%----------------------------------------------------------------------------------------

\Large{\section{Gramática}}

\S\ 1. Los pronombres personales tienen dos formas en el estonio, una forma larga que es utilizada cuando se quiere enfatizar el pronombre y una forma corta que se utiliza cuando particularmente no se desea enfatizar el pronombre en la oración.\\

\begin{tabular}{ l l c l l }
	\textbf{mina — ma} & `yo'		& &	\textbf{meie — me} 	& `nosotros' \\
	\textbf{sina — sa} & `tú'		& &	\textbf{teie — te}	& `ustedes' \\
	\textbf{tema — ta} & `él/ella'	& &	\textbf{nemad — nad}&`ellos/ellas'
\end{tabular}\\ \bigskip

\S\ 2. El estonio carece de palabras distintas para `él, ella'. La palabra \textbf{tema} se utiliza para los dos. Lo que se quiere decir en realidad sólo puede ser determinado a partir del contexto en el que se utiliza la palabra.\\

\Large{\section{Tiempo Presente}}

\S\ 3. Los verbos se conjugan por persona y tiempo. El siguiente es un ejemplo de un verbo en tiempo presente:\\

\begin{tabular}{ l l l }
	1\textordmasculine\ persona singular: 	& \textbf{mina tule\underline{n}} 		& `Yo vengo' \\
	2\textordmasculine\ persona singular: 	& \textbf{sina tule\underline{d}} 		& `Tú vienes' \\
	3\textordmasculine\ persona singular: 	& \textbf{tema tule\underline{b}} 		& `Él/Ella viene' \\
	 & & \\
	1\textordmasculine\ persona plural:		& \textbf{meie tule\underline{me}} 		& `Nosotros venimos' \\
	2\textordmasculine\ persona plural:		& \textbf{teie tule\underline{te}} 		& `Ustedes vienen' \\
	3\textordmasculine\ persona plural:		& \textbf{nemad tule\underline{vad}} 	& `Ellos/Ellas vienen'
\end{tabular}\\ \bigskip

Nota: Al igual que en el español, los pronombres personales a menudo pueden ser omitidos, ya que la propia forma verbal es suficiente para indicar qué persona es el sujeto: \textbf{tulen} `vengo', \textbf{tuleme} `venimos'.\\

\S\ 4. Toda forma en tiempo presente consiste en una raíz, que se mantiene sin cambios para todas las personas (\eg, \textbf{tule-}), y diversas terminaciones personales, para cada persona en singular y plural. Un verbo en estonio tiene las siguientes terminaciones en el tiempo presente:\\

\begin{tabular}{ l l c l l }
	1\textordmasculine\ persona singular: & \textbf{-n}	& & 1\textordmasculine\ persona plural: & \textbf{-me} \\
	2\textordmasculine\ persona singular: & \textbf{-d}	& & 2\textordmasculine\ persona plural: & \textbf{-te} \\
	3\textordmasculine\ persona singular: & \textbf{-b}	& & 3\textordmasculine\ persona plural: & \textbf{-vad}
\end{tabular}\\ \bigskip

Si usted conoce una de las formas en tiempo presente, puede construir el resto a partir de ella. Usted puede tomar, por ejemplo, la primera persona singular (la cual es siempre dada en nuestra lista de palabras), botar la \textbf{-n} final y así conseguir la raíz del presente. Luego se agregan las terminaciones listadas arriba para obtener las formas en tiempo presente restantes.\\

Ejemplos: palu/n `(yo) ruego', räägi/n `(yo) hablo', õpi/n `(yo) aprendo, estudio'\\

\begin{tabular}{ l l l l l }
	\emph{Singular}	& mina (ma)		& palu/n	& räägi/n	& õpi/n \\
					& sina (sa)		& palu/d	& räägi/d	& õpi/d \\
					& tema (ta)		& palu/b	& räägi/b	& õpi/b \\
					& & & & \\
	\emph{Plural}	& meie (me)		& palu/me	& räägi/me	& õpi/me \\
					& teie (te)		& palu/te	& räägi/te	& õpi/te \\
					& nemad (nad)	& palu/vad	& räägi/vad	& õpi/vad 
\end{tabular}\\ \bigskip

\S\ 5. El tiempo presente del verbo \textbf{ole/n} `ser, estar' es irregular en tercera persona singular y plural:\\

\begin{tabular}{ l l l l }
	mina olen 			& `yo soy, estoy'		& meie oleme 				& `nosotros somos, estamos' \\
	sina oled 			& `tú eres, estás' 		& teie olete 				& `ustedes son, están' \\
	tema \textbf{on} 	& `él/ella es, está'	& nemad \textbf{on} 		& `ellos/ellas son, están'
\end{tabular}\\ \bigskip

\S\ 6. El estonio carece de un tiempo futuro nítido. El tiempo presente puede ser usado para indicar el futuro también. Tanto el presente como el futuro sólo puede ser determinado a partir del contexto en el que aparece la palabra.\\

\begin{tabular}{ l l l l }
	ma tulen praegu & `yo vengo de inmediato' \\
	ma tulen homme 	& `yo vendré mañana'
\end{tabular}\\ \bigskip

\S\ 7. Al igual que en el español y otros idiomas, la segunda persona plural se usa para mostrar respeto o distancia social al dirigirse a una persona que no conozca o que no llama por su primer nombre.\\

Ejemplo: Millal te tulete, härra Palm? `¿Cuándo vendrá, Sr. Palm?'\\

\section*{\Large{Texto}}

Mina olen améeriklane. Ma elan Améerikas. Sina oled eestlane. Sa elad ka Ameerikas. Mina olen kodus. Sina oled ka kodus. Sa kirjutad. Tema on siin. Meie tuleme homme. Me oleme täna kodus. Teie tulete ja räägite. Te räägite eesti keelt. Nemad on seal. Nad räägivad inglise keelt.\\

Olen siin. Õpin. Ma õpin eesti keelt. Tema õpib ka eesti keelt. Oleme kodus. Meie kirjutame. Nemad õpivad. Teie elate Ameerikas. Te räägite hästi inglise keelt. Täname. Sina kirjutad hästi. Tänan väga. Palun.\\

\section*{\Large{Vocabulario}}

\begin{tabular}{ l l }
	Améerika 							& América \\
	Ameerikas 							& en América \\
	améeriklane 						& un Americano \\
	eesti keel 							& el idioma estonio \\
	eestlane 							& un estonio \\
	ela/n 								& (yo) vivo \\
	homme 								& mañana \\
	hästi 								& bien \\
	inglise keel 						& el idioma inglés \\
	ja 									& y \\
	ka 									& también \\
	kiijuta/n 							& (yo) escribo \\
	kodus 								& en la casa \\
	palu/n 								& (yo) ruego; por favor; de nada \\
	räägi/n	\rule{1cm}{0.4pt} keelt 	& (yo) hablo el idioma \rule{1cm}{0.4pt} \\
	seal 								& ahí, allí \\
	siin 								& aquí \\
	tule/n 								& (yo) vengo \\
	täna 								& hoy \\
	täna/n 								& (yo) agradezco; gracias\\
	väga 								& mucho \\
	õpi/n 								& (yo) aprendo, estudio
\end{tabular}\\ \bigskip

\section*{\Large{Ejercicios}}

\begin{enumerate}
	\item \emph{Conjugar los siguientes verbos en el tiempo presente en estonio:} agradecer, venir, hablar, rogar, estudiar, ser.
	\item \emph{Traducir al estonio:} Yo hablo. Nosotros estamos aquí. Él viene mañana. Ustedes hablan bien. Ella está ahí. Tú está en la casa. Ellos están aquí y están estudiando. Ustedes también están aquí. Nosotros hablamos. Ellos vienen hoy. Yo ruego. Usted está viniendo. Gracias.
\end{enumerate}

\section*{\Large{Expresiones de saludo y agradecimiento}}

\begin{tabular}{ l p{8cm} }
	Tere!					& ¡Hola! \\
	Tervist!				& ¡Buen día! ¡Saludos! \\
	Tere tulemast! 			& ¡Bienvenido/a! \\
	Palun!					& \small{[Cuando alguien pide algo y uno acepta]} ¡Por favor! ¡Aquí tiene! \small{[En respuesta a un gracias]} ¡De nada! \\
	Tänan!					& ¡Gracias! Te agradezco \\
	Tänan väga! 			& ¡Muchas gracias! \\
	Aitäh!					& ¡Gracias!
\end{tabular}\\ \bigskip

\section*{\Large{Respuesta a los ejercicios}}

\begin{enumerate}
\item 
\begin{tabular}{ l l l l l l l }
	(mina)	& tänan		& tulen		& räägin	& palun		& õpin		& olen \\
	(sina)	& tänad		& tuled		& räägid	& palud		& õpid		& oled \\
	(tema)	& tänab		& tuleb		& räägib	& palub		& õpib		& on \\
	(meie)	& täname	& tuleme	& räägime	& palume	& õpime		& oleme \\
	(teie)	& tänate	& tulete	& räägite	& palute	& õpite		& olete \\
	(nemad)	& tänavad	& tulevad	& räägivad	& paluvad	& õpivad	& on 
\end{tabular}\\ \bigskip
\item Mina räägin. Meie oleme siin. Tema tuleb homme. Teie räägite hästi. Tema on seal. Sina oled kodus. Nemad on siin ja õpivad. Teie olete ka siin. Meie räägime. Nemad tulevad täna. (Ma) palun. Sina tuled. (Ma) tänan.\\

[La versión corta de los pronombres (ma, sa, ta, me, te, nad) también estarían correctas, si es que no desea enfatizar en el pronombre como sujeto de cada oración.]
\end{enumerate}

%----------------------------------------------------------------------------------------