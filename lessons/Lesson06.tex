% Lesson 6

\chapter{Sexta Lección} % Chapter title

\label{ch:lesson06} % For referencing the chapter elsewhere, use \autoref{ch:examples} 

%----------------------------------------------------------------------------------------


% =====================
% 		GRAMATICA
% =====================
\Large{\section*{Gramática}}

\bemph{\subsection*{Tiempo Presente Condicional}}

\S\ 27. El presente condicional, que corresponde a las expresiones con terminación `-ría' en español, se construye en el estonio con la raíz del tiempo presente, seguido de \bemph{-ksi-} y la terminación final en tiempo presente en la primera y segunda persona singular/plural. La tercera persona del singular/plural no sigue este patrón. \\

\begin{tabular}{ l l l }
	 					& \emph{Presente} 					& \emph{Presente Condicional} \\
	\emph{Singular} 	& 1. ma \bemph{taha/n} `Yo quiero' 	& \bemph{taha/ksi/n} `Yo querría' \\
	 					& 2. sa \bemph{taha/d} 				& \bemph{taha/ksi/d} \\
	 					& 3. ta \bemph{taha/b} 				& \bemph{taha/ks/-} \\
	 					& & \\
	\emph{Plural} 		& 1. me \bemph{taha/me} 			& \bemph{taha/ksi/me} \\
	 					& 2. te \bemph{taha/te} 			& \bemph{taha/ksi/te} \\
	 					& 3. nad \bemph{taha/vad} 			& \bemph{taha/ksi/d}
\end{tabular}
\bigskip

Ejemplo: Ma tahaksin minna koju. `Querría (Me gustaría) ir a casa.' \\

\S\ 28. La forma negativa del condicional consiste en \bemph{ei + tercera persona singular} del verbo, para todas las personas. \\

\begin{tabular}{ l l }
	ma \bemph{ei tahaks} `Yo no querría' 	& me \bemph{ei tahaks} \\
	sa \bemph{ei tahaks} 					& te \bemph{ei tahaks} \\
	ta \bemph{ei tahaks} 					& nad \bemph{ei tahaks}
\end{tabular}
\bigskip

Ejemplo: Me \bemph{ei tahaks} minna koju. `No querríamos (No nos gustaría) ir a casa.' \\

Nota: Esta forma es aceptada incluso en el caso afirmativo, sobre todo en estonio hablado. En este caso, el elemento negativo `ei' se descarta. \\

\begin{center}
Ma \bemph{tahaks} [en vez de \bemph{tahaksin}] minna koju. \\
`Querría (Me gustaría) ir a casa.'
\end{center}
\bigskip

\S\ 29. En resumen, la raíz presente del verbo se utiliza para construir tanto las formas positivas como negativas de: \\

\begin{enumerate}
	\item el tiempo presente (ver \autoref{ch:lesson01})
	\item el imperativo para la segunda persona singular (ver \autoref{ch:lesson02})
	\item el tiempo condicional
\end{enumerate}
\bigskip

Por ejemplo: \\

\begin{tabular}{ l | l l }
								& \bemph{tule/n} 		& \bemph{ei tule} \\
								& `(Yo) vengo			& `(Yo) no vengo \\ 
								& & \\
	\emph{raíz del presente}	& \bemph{tule!} 		& \bemph{ära tule!} \\
	\bemph{tule-} `venir'		& `¡Ven!'				& `¡No vengas!' \\
								& & \\
								& \bemph{tule/ksi/n} 	& \bemph{ei tule/ks} \\
								& `(Yo) vendría'		& `(Yo) no vendría'
\end{tabular}
\bigskip

\S\ 30. El infinitivo (`hablar', `ir', etc. en español) a menudo tiene otra raíz que la del presente en estonio. Dado que la raíz del infinitivo es una de las formas básicas en la conjugación de un verbo en estonio (ver \autoref{ch:lesson21}), es necesario conocer tanto la raíz del infinitivo como del presente. En todos los diccionarios, el verbo siempre aparece como infinitivo (con la terminación -ma), y así será también en nuestro glosario. El infinitivo-ma es seguido por la forma en tiempo presente. Por ejemplo: \\

\begin{tabular}{ l l }
	\bemph{lugema, loe/n} 	& `leer, (Yo) leo' \\
	\bemph{lubama, luba/n} 	& `permitir, (Yo) permito' \\
	\bemph{tahtma, taha/n} 	& `querer, (Yo) quiero'
\end{tabular}

% ==================
% 		TEXTO
% ==================
\bigskip
\Large{\section*{Texto}}

Ma lähen jalutama, kui ema lubab. Ma \emph{läheksin} jalutama, kui ema \emph{lubaks}. Ma ei lähe sinna, jui sa ka lubad. Ma \emph{ei läheks} sinna, kui sa ka \emph{lubaksid}. Olen siin, kui sa tuled. \emph{Oleksin} väga rõõmus, kui sa homme siia \emph{tuleksid}. Oleks tore, kui sõber ka \emph{tuleks}. \\

Õpetaja küsib ja õpilane vastab. Kas ta oskab? Õpilane \emph{vastaks}, kui ta ainult \emph{oskaks}. Nemad istuvad, aga sina seisad. Nemad \emph{istuksid}, kui sa \emph{paluksid}. Me tahame, et te laulate. Me \emph{tahaksime}, et te \emph{laulaksite}. Kas ma tohin? Ma \emph{laulaksin}, kui ma \emph{tohiksin}. Ma \emph{tahaksin}, et sa \emph{õpiksid} hästi.	\\

Ma \emph{ei tahaks}, et sa ainult lamad ja puhkad. Sa saad, kui sa soovid. Kui sa ilusti \emph{paluksid}, siis sa \emph{saaksid}. Tee nii, nagu isa ütleb. \emph{Sooviksin}, et sa \emph{teeksid} nii, nagu ma ütlen. Nad \emph{ei teeks} nii, kui nad \emph{oleksid} kodus. \\

\begin{center}
Oleksin laululind, \\
kannaksid tiivad mind!
\end{center}
\bigskip

\noindent
Istun üksi toas. Keegi koputab. \\
-- Tee uks lahti! hüüab üks hääl. \\
-- Üks silmapilk. Tulen kohe ... Ah, sina oled! Tere! Astu sisse. Ole hea, pane uks kinni. Istu. \\
-- Tänan väga. Kas lubad, ma suitsetan? küsib sõber. \\
-- Palun väga. Luba mulle ka üks suits. \\
-- Säh, siin on karp. Võta ise üks sigarett. Siin on tikud, palun. 

% =======================
% 		VOCABULARIO
% =======================
\bigskip
\Large{\section*{Vocabulario}}

\begin{tabular}{ l l }
	ainult					& sólo [\emph{adv.}] \\
	aken					& ventana \\
	astuma, astu/n			& dar un paso, (Yo) doy un paso \\
	avatud					& abierto [\emph{adj.}] \\
	hääl					& voz \\
	hüüdma,	hüüa/n			& gritar, llamar; (Yo) grito, llamo \\
	ilusti					& bellamente, hermosamente \\
	kandma, kanna/n			& llevar, acarrear, (Yo) llevo, acarreo \\
	karp					& caja pequeña  \\
	keegi					& alguien \\
	kinni					& cerrado \\
	koputama, koputa/n		& llamar (a la puerta), (Yo) llamo a la puerta \\
	kui ... ka				& incluso si ... \\
	lahti					& abierto [\emph{adj.}] \\
	laululind				& ave cantora \\
	lind					& ave, pájaro \\
	lubama, luba/n			& permitir, (Yo) permito \\
	mind					& mí \\
	oskama, oska/n			& ser capaz, saber, (Yo) puedo \\
	pane/n kinni			& (Yo) cierro \\
	rõõmus					& feliz, contento \\
	saama, saa/n			& obtener, ser capaz; (Yo) puedo, obtengo \\
	sigarett				& cigarrillo \\
	silmapilk				& momento, `en un abrir y cerrar de ojos' \\
	sisse					& adentro  \\
	soovima, soovi/n		& desear, (Yo) deseo \\
	suits					& humo, cigarrillo \\
	suitsetama, suitseta/n	& fumar, (Yo) fumo \\
	suletud					& cerrado [\emph{adj.}] \\
	säh!					& ¡aquí tiene! ¡tome! \\
	tahtma, taha/n			& querer, (Yo) quiero  \\
	tee/n lahti				& (Yo) abro [\emph{v.}] \\
	tiivad					& alas \\
	tikud					& fósforos \\
	toas					& e la habitación  \\
	tohtima, tohi/n			& poder, tener permiso, (Yo) puedo, tengo permiso \\
	uks						& puerta \\
	üksi					& solo [\emph{adj.}]
\end{tabular}

% ======================
% 		EJERCICIOS
% ======================
\bigskip
\Large{\section*{Ejercicios}}

\begin{enumerate}
	\item \emph{Conjugado en tiempo presente condicional:} soovin `yo deseo', ütlen `yo digo', laulan `yo canto', lähen `yo voy', tulen `yo vengo', võtan `yo tomo', palun `yo ruego'.

	\item \emph{Traducir al estonio:} Yo cantaría, si pudiera. A él le gustaría que fueras allí. Nosotros vendríamos, si tuviéramos permiso. ¡Dame un nuevo libro! El tiempo está hermoso hoy. ¡Llame a la hermana para aquí! ¡Dame una caja pequeña! ¿Tú fumas mucho? ¿Ustedes lo permiten?

	\item \emph{Traducir al español:} Tee uks lahti [=Ava uks]! Uks on lahti [= Uks on avatud]. Pane aken kinni! Aken on kinni [= Aken on suletud]. Ava raamat [= Tee raamat lahti]! Pane raamat kinni!
\end{enumerate}

% ============================
% 		EXPRESIONES DE ...
% ============================
\bigskip
\Large{\section*{Expresiones de Invitación}}

\begin{tabular}{ p{6cm} p{6cm} }
	\bemph{Oleksin väga rõõmus, kui sa tuleksid/te tuleksite.}	& Yo estaría muy contento si vinieras/vinieses. \\
	& \\
	\bemph{Tuleksin heameelega, kuid kahjuks olen täna kinni.}	& Iría encantado, pero lamentablemente hoy estoy ocupado. \\
	& \\
	\bemph{Kahjuks ma ei saa.}									& Por desgracia no puedo. \\
	& \\
	\bemph{Oleksin sulle/teile väga tänulik.}					& (Te/Le) estaría muy agradecido.  \\
	& \\
	\bemph{Kas oleks võimalik...?}								& ¿Sería posible ...? \\
	& \\
	\bemph{Paluksin...}											& Me gustaría pedir ... [\emph{lit.:} mendigar] \\
	& \\
	\bemph{Kas sa suitsetad? Kas te suitsetate? }				& ¿Tú fumas? ¿Usted fuma? \\
	& \\
	\bemph{Ei, tänan. Ma ei soovi praegu.}						& No, gracias. Yo no quiero en este momento. \\
	& \\
	\bemph{Ma ei suitseta.}										& Yo no fumo. \\
	& \\
	\bemph{Kas tohin? Kas lubate?}								& ¿Puedo? ¿usted lo permite? \\
	& \\
	\bemph{Luba mulle ûks suits/tikk!}							& ¡Permítame un cigarrillo/fósforo!  \\
	& \\
	\bemph{Ole hea [Palun], võta/võtke üks sigarett!}			& Por favor, ¡ten/tenga un cigarrillo! \\
	& \\
	\bemph{Siin on tikud.}										& Aquí están los fósforos. 
\end{tabular}

% ======================================
% 		RESPUESTA A LOS EJERCICIOS
% ======================================
\bigskip
\Large{\section*{Respuesta a los ejercicios}}

\begin{enumerate}
	\item \emph{Conjugación:} \\

	\begin{tabular}{ l l l l l }
		ma	& sooviksin 	& ütleksin 	& laulaksin 	& läheksin \\
		sa	& sooviksid 	& ütleksid 	& laulaksid 	& läheksid \\
		ta	& sooviks 		& ütleks 	& laulaks 		& läheks \\	
		me	& sooviksime 	& ütleksime & laulaksime 	& läheksime \\
		te	& sooviksite 	& ütleksite & laulaksite 	& läheksite \\
		nad	& sooviksid 	& ütleksid 	& laulaksid 	& läheksid \\
		& & & & \\ 
		ma 	& tuleksin 		& võtaksin	& paluksin 		& \\
		sa 	& tuleksid 		& võtaksid 	& paluksid 		& \\
		ta 	& tuleks 		& võtaks 	& paluks 		& \\
		me 	& tuleksime 	& võtaksime & paluksime 	& \\
		te 	& tuleksite 	& võtaksite & paluksite 	& \\
		nad & tuleksid 		& võtaksid 	& paluksid 		&  
	\end{tabular}

	\item Ma laulaksin, kui ma oskaksin. Ta tahaks, et sa läheksid sinna. Me tuleksime, kui me tohiksime. Anna mulle (üks) uus raamat! Täna on ilus ilm. Kutsu õde siia! Anna mulle (üks) väike karp! Kas sa suitsetad palju? Kas (te) lubate?

	\item Abra la puerta. La puerta está abierta. Cierre la ventana. La ventana está cerrada. Abra el libro. Cierre el libro.
\end{enumerate}

%----------------------------------------------------------------------------------------