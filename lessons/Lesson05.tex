% Lesson 5

\chapter{Quinta Lección} % Chapter title

\label{ch:lesson05} % For referencing the chapter elsewhere, use \autoref{ch:examples} 

%----------------------------------------------------------------------------------------


% =====================
% 		GRAMATICA
% =====================
\Large{\section*{Gramática}}

\S\ 24. Después de una orden, el llamado objeto definido u objeto total se encuentra en el caso nominativo. Un objeto es "total" si el mismo  está implicado en la acción (véase el \autoref{ch:lesson28}). Ejemplos:

\begin{center}
\begin{tabular}{ l l }
	Too \bemph{raama}t siia!		&	`¡Trae el libro (aquí)!' \\
	Vii \bemph{laps} koju!			&	`¡Lleva al niño a casa!' \\
	Kutsu \bemph{vend} siia!		&	`¡Llama al hermano aquí!' \\
	Anna mulle \bemph{üks dollar}!	&	`¡Dame (a mí) un dolar!' \\
	Võta see \bemph{ajaleht}!		&	`¡Toma ese periódico!' 
\end{tabular}
\end{center}
\bigskip

\S\ 25. Construcciones impersonales del tipo `está caluroso' se expresan en estonio sólo por el verbo en tercera persona del singular, al igual que en español. \\

\begin{center}
\begin{tabular}{ l l }
	\bemph{On} võimaik, et ...		&	`Es posible que ...' \\
	Täna \bemph{on} ilus ilm.		&	`Hoy es un día hermoso.' \\
	Toas \bemph{on} soe.			&	`En la habitación está caluroso.' \\
	Kuidas \bemph{läheb}?			&	`¿Cómo estás?' [\emph{lit.:} ¿Cómo va?] \\
	\bemph{Sajab}.					&	`Llueve.' \\
	Mind \bemph{huvitab}, kas ...	&	`Me interesa, que ...' 
\end{tabular}
\end{center}
\bigskip

En ciertos casos, incluso el verbo puede omitirse: \\

\begin{center}
\begin{tabular}{ l l }
	Väga \bemph{võimalik}, et ...	& `Es muy probable que ...' \\
	\bemph{Huvitav}, kas ...		& `Sería interesante que ...' \\
	\bemph{Imelik}, et ...			& `Es extraño que ...'
\end{tabular}
\end{center}
\bigskip

\S\ 26. Números 0 - 10 \\

\begin{center}
\begin{tabular}{ l c l }
	0	null	& &	6 kuus \\
	1	üks		& &	7 seitse \\
	2	kaks	& &	8 kaheksa \\
	3	kolm	& &	9 üheksa \\
	4	neli	& &	10 kümme \\
	5	viis	& & 
\end{tabular}
\end{center}
\bigskip

% ==================
% 		TEXTO
% ==================
\Large{\section*{Texto}}

Võta raamat ja tule siia. Ava raamat ja loe! Kas see on huvitav raamat? See on õpik! \\
Võta see sulepea ja kirjuta. Kirjuta üks kiri! Saada kiri isale. Anna sulepea mulle tagasi. \\

Ole hea, too pliiats siia! Võta pliiats ja joonista. Joonista üks pilt. See on ilus pilt! Kingi see pilt mulle! \\

Osta homme uus õpik. Too õpik kaasa, kui sa tuled. Kutsu sõber ka kaasa. Huvitav, kas ta tuleb? Kui on halb ilm, siis istume toas ja õpime. Kui aga ilm on ilus, siis läheme jalutama. \\
Täna sa töötad, aga homme puhkad. Sa õpid hästi, sa oled hea õpilane. Tänan, kuid teie õpetate hästi. Te olete hea õpetaja. \\
Väike tüdruk laulab. See on väga ilus laul. Laula veel üks laul! \\

Kui palju on seitse ja kolm? Seitse ja kolm on kümme. Kui palju on kaks ja viis? Kaks ja viis on seitse. Kui palju on üks pluss neli? Üks pluss neli on viis. Üheksa miinus kuus on kolm. \\

\noindent
-- Ütle, palun, mis arv see on: 5? \\
-- See on viis. \\
-- Õige! Aga mis arv see on: 7? \\
-- Kaheksa? \\ 
-- Vale! See on seitse. Õpi veel! \\

\begin{center}
Üheksa korda mõtle, üks kord ütle. (Vanasõna)
\end{center}

% =======================
% 		VOCABULARIO
% =======================
\Large{\section*{Vocabulario}}

\begin{tabular}{ l l }
anna/n		&	(Yo) doy \\
arv			&	número \\
ava/n		&	(Yo) abro \\
halb		&	malo \\
huvitav		&	interesante \\
ilm			&	tiempo, clima \\
isale		&	al padre \\
jaluta/n	&	(Yo) camino, paseo \\
joonista/n	&	(Yo) dibujo \\
kaasa		&	consigo \\
kingi/n		&	(Yo) regalo \\
kiri		&	carta \\
kord		&	vez, ocasión \\
korda		&	veces, ocasiones \\
kui palju	&	Cuánto \\
kutsu/n		&	(Yo) invito, llamo \\
laul		&	canción \\
laula/n		&	(Yo) canto \\
miinus		&	menos \\
mulle		&	para mí \\
osta/n		&	(Yo) compro \\
palju		&	mucho \\
pilt		&	imagen, fotografía \\
pliiats		&	lápiz \\
pluss		&	más, suma \\
raamat		&	libro \\
saada/n		&	(Yo) envío \\
sulepea		&	pluma (para escribir) \\
toas		&	en la habitación \\
too/n		&	(Yo) traigo, llevo \\
tööta/n		&	(Yo) trabajo \\
uus			& 	nuevo \\
vale		&	incorrecto, malo \\
võta/n		&	(Yo) tomo \\
õige		&	correcto, bien \\
õpetaja		&	profesor \\
õpeta/n		&	(Yo) enseño \\
õpik		&	manual \\
õpilane		&	estudiante
\end{tabular}
\bigskip

% ======================
% 		EJERCICIOS
% ======================
\Large{\section*{Ejercicios}}

\begin{enumerate}
	\item \emph{Traducir al estonio:} El padre es un profesor. El maestro enseña. El hijo es un estudiante. El estudiante estudia. ¡Dame este libro! Por favor, abre el libro. Canta una canción. Hay buen tiempo hoy. No voy a hacer nada hoy. Descansaremos hoy, pero mañana vamos a trabajar. ¿A dónde van? Estamos caminando a casa. Compra un nuevo libro. Lleva el libro contigo, cuando vengas.

	\item \emph{Diga en estonio:} ¿Cuánto es dos y cuatro? 1+9 = 10, 2+6 = 8, 3+4 = 7, 5+5 = 10, 8-7 = 1, 9-6 = 3, 2-2 = 0, 6-5 = 1.

	\item \emph{Traduzca las siguientes frases al español:} \\
		\begin{tabular}{ l l }
			tõusen püsti 	& saad aru \\
			vaatame pealt 	& näete välja  
		\end{tabular}

	\item \emph{Traduzca las siguientes frases al español:} Poiss \bemph{ja} tüdruk. Suur \bemph{või} väike? \bemph{Kas} jah \bemph{või} ei? Ma näen, \bemph{et} ... Tüdruk istub, \bemph{aga} poiss seisab. See on \bemph{nii} ilus! Õde ei ole \bemph{nii} suur \bemph{kui} vend.\bemph{Kui} sa tuled, võta sõber \bemph{ka} kaasa!
\end{enumerate}

% ============================
% 		EXPRESIONES DE ...
% ============================
\Large{\section*{Expresiones de Evaluación}}

\begin{tabular}{ l l }
	\bemph{Kas sa oled rahul?}						& ¿Estás satisfecho? [\emph{lit.:} en paz] \\
	\bemph{Jah, täiesti!}							& Sí, completamente. \\
	\bemph{Suur [= Palju] tänu, kõik on korras.}	& Muchas gracias, todo está en orden. \\
	\bemph{On(s) see tõsi?}							& ¿Es cierto? \\
	\bemph{See on sulatõsi!}						& ¡Es la pura verdad! \\
	\bemph{See on puha vale!}						& ¡Es una mentira! \\
	\bemph{Sa eksid! Te eksite!}					& ¡Te equivocas! ¡Se equivoca (usted)! \\
	\bemph{Laula üks laul!}							& ¡Canta una canción! \\
	\bemph{Ma ei oska.}								& No puedo. (No sé cómo.) \\
	\bemph{Ah nii?!}								& ¡¿Ah, sí?! 
\end{tabular}

% ======================================
% 		RESPUESTA A LOS EJERCICIOS
% ======================================
\Large{\section*{Respuesta a los ejercicios}}

\begin{enumerate}
	\item Isa on õpetaja. Õpetaja õpetab. Poeg on õpilane. Õpilane õpib. Anna mulle see raamat! Palun [= Ole hea], ava raamat! Laula üks laul! Täna on ilus ilm. Ma ei tee täna mitte midagi. Täna me puhkame, aga [= kuid] homme me töötame. Kuhu te lähete? Me jalutame koju. Osta (üks) uus raamat. Too raamat kaasa, kui (sa) tuled.

	\item Kui palju on kaks ja neli? Üks ja [= pluss] üheksa on kümme, kaks ja kuus on kaheksa, kolm ja neli on seitse, viis ja viis on kümme, kaheksa miinus seitse on üks, üheksa miinus kuus on kolm, kaks miinus kaks on null, kuus miinus viis on üks

	\item 
		\begin{tabular}{ l l }
			Me levanto 			& tú entiendes \\
			miramos, observamos & ustedes parecen, se ven como  
		\end{tabular}

	\item Un niño y una niña. ¿Grande o pequeño? ¿Sí o no? Veo que ... La niña está sentada, pero el niño está parado. ¡Esto es tan hermoso! La hermana no es tan grande como el hermano. Cuando vengas, ¡trae al amigo contigo también!.
\end{enumerate}

%----------------------------------------------------------------------------------------