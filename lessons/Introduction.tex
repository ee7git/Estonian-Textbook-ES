% ======================================
%
% 				Introduction
%
% ======================================

\pdfbookmark[1]{Introducción}{Introducción} % Bookmark name visible in a PDF viewer

\chapter*{Introducción} 

%----------------------------------------------------------------------------------------

\begin{enumerate}
	\item El alfabeto estonio básico consta de 23 letras en el siguiente orden:
	\begin{center}
	\begin{otherlanguage}{estonian}
		\bemph{a b c d e g h i j k l m n o p r s t u v õ ä ö ü}
	\end{otherlanguage}
	\end{center}

	Existen otras 9 letras que aparecen en las palabras extranjeras. Las letras \bemph{c q w x y} se encuentran sólo en los nombres extranjeros, como César, Don Quijote, Xantippe, Nueva York. Las letras \bemph{f š z ž} se encuentran en las palabras adoptadas de otros idiomas: film, šokolaad, zooloog, žurnaal.\\ 
	Finalmente, el orden del alfabeto completo es el siguiente:
	\begin{center}
	\begin{otherlanguage}{estonian}
		\bemph{a b c d e f g h i j k l m n o p q r s š z ž t u v w õ ä ö ü x y}
	\end{otherlanguage}
	\end{center}

	\item La mayoría de las letras se pronuncian de manera bastante similar al español. Sin embargo, tenga en cuenta lo siguiente:

	\section*{\Large{Vocales}}

	Las vocales \bemph{a e i o u} se pronuncian exactamente igual que en español, sin embargo la fonética de las vocales \bemph{ä õ ö ü} puede presentar una gran dificultad.\\

	La vocal \bemph{ä} se genera al intentar pronunciar las vocales \bemph{a} y \bemph{e} al mismo tiempo: \bemph{\ae}.\\

	La vocal \bemph{ö} se genera pronunciando la vocal \bemph{e} pero redondeando los labios como si se fuera a pronunciar la vocal \bemph{o}.\\

	La vocal \bemph{ü} se genera pronunciando la vocal \bemph{i} pero redondeando los labios como si se fuera a pronunciar la vocal \bemph{u}.\\

	La vocal \bemph{õ} se genera pronunciando la vocal \bemph{o} pero sin redondear los labios, como si se fuera a pronunciar la vocal \bemph{e}.\\

	Por supuesto lo más recomendable, más que leer una descripción, es escuchar el audio de las pronunciaciones de las vocales. El siguiente \href{http://www.youtube.com/watch?v=GfxrR45yA6I}{video}\footnote{http://www.youtube.com/watch?v=GfxrR45yA6I}, cortesía de \href{http://www.engetranslations.ee/airien.htm}{Airi Enge}\footnote{http://www.engetranslations.ee/airien.htm}, presenta de forma clara y precisa todas las vocales del estonio.\\

 	\section*{\Large{Consonantes}}

 	La constante \bemph{z} se pronuncia igual que la palabra inglesa `zoo'.\\

 	La constante \bemph{j} se pronuncia como la letra \bemph{y} en la palabra `maya'.\\

 	La constante \bemph{h} posee diferentes casos. Si está al principio de una palabra el sonido es muy débil, casi un silencio. Si está en medio de dos vocales se pronuncia como la \bemph{j} en la interjección coloquial `¡Ajá!', y si se encuentra antes de una consonante o al final de una palabra su pronunciación se torna más como una exhalación muy suave, haciendo amagos en formar una \bemph{j}.\\

 	Las consonantes \bemph{b d g} son sordas y ligeramente más suaves que las letras del español \bemph{p t k}, respectivamente.\\

 	Las consonantes \bemph{p t k} también son sordas, pero ligeramente más fuertes que las del español. Un sonido más duradero y ligeramente más potente se da en el caso de tener consonantes dobles \bemph{pp tt kk}\\

 	La consonante \bemph{š} equivale a la combinación \bemph{sh} en inglés, como en la palabra `english'.\\

 	La consonante \bemph{ž} equivale a la letra \bemph{s} en la palabra inglesa `treasure', o como la \bemph{j} de la palabra francesa `jour'.\\

 	Las consonantes \bemph{l r s m n v} se pronuncian exactamente igual que en español.\\

 	Las consonantes \bemph{f c q w x y} dependen del idioma del cual provienen.\\

 	\item En el estonio hay muchos diptongos o combinaciones de dos vocales que forman parte de la misma sílaba. Estos incluyen \bemph{ae ai ao au ea ei eo iu oa oe oi õe äe} y así sucesivamente. Cada una de las dos vocales se pronuncia con claridad, pero no como si estuvieran en sílabas distintas.\\

 	Ejemplos: laud, laev, hea, loen, õun, õed, käed.\\

 	\item La acentuación en el estonio, a diferencia del español, se encuentra normalmente en la primera sílaba. Hay, sin embargo, algunas excepciones como \foreignlanguage{estonian}{ai\bemph{täh} `gracias', sõb\bemph{ran}na `amiga', üle\bemph{üld}se `sobre todo'}. El acento de muchas palabras adoptadas de otros idiomas se ha mantenido: e\bemph{lek}ter, ide\bemph{aal}, pro\bemph{fes}sor. Cuando la marca \textasciiacute\ es usada en el texto sobre una vocal (como en elékter, ideáal, proféssor), indica que el acento está sobre esa sílaba, sin embargo esta marca no es parte de la ortografía normal.\\

 	\item La ortografía del estonio es fundamentalmente fonética, lo que significa que las palabras están escritas como suenan. Como una regla básica, letras solas significan sonidos cortos y letras dobles indican sonidos largos.\\

 	La pronunciación de las vocales individuales siempre es muy corto (primer grado), en contraste con las interminables vocales de palabras en inglés como `go', `at', `find'.\\

 	\item Vocales dobles, consonantes dobles y diptongos son largos (segundo grado) o muy largos (tercer grado).\\

 	Cada vocal y consonante en el estonio puede tener por lo tanto tres diferentes largos o grados:\\

 	\begin{tabular}{ r l l c l l}
 		1\textordmasculine\ grado: & s\bemph{a}da 					& `cien' 		& & li\bemph{n}a 					& `mantel, sábana' \\
 		2\textordmasculine\ grado: & s\bemph{aa}da 					& `¡envía!' 	& & li\bemph{nn}a 					& `ciudades' \\
 		3\textordmasculine\ grado: & \textasciigrave s\bemph{aa}da 	& `obtener' 	& & \textasciigrave li\bemph{nn}a 	& `hacia la ciudad' 
 	\end{tabular}

 	El tercer grado es notablemente más largo que los sonidos correspondientes al español o el inglés.\\

	En la transcripción fonética, el tercero grado se indican con una \textasciigrave\ antes de la sílaba. Esto no se utiliza en el lenguaje escrito, pero se utiliza en los diccionarios y listas de palabras en los casos en que la longitud afecta el significado. Por ejemplo: \textasciigrave Kooli (pronunciado como si hubieran 3 vocales - koooli) significa `a la escuela', en comparación con Kooli (pronunciado con sólo dos vocales) que significa `de la escuela'.\\

	\item En algunas palabras, las consonantes \bemph{l n s t} se suavizan o palatalizan con un ligero \bemph{i} o \bemph{j} (la \bemph{j} estonia) antes de la consonante.\\

	\begin{tabular}{ l l}
	Ejemplos: 	& palk (pal\textquotesingle k) `tronco, viga' \\
				& tund (tun\textquotesingle d) `hora' \\
				& kott (kot\textquotesingle t) `bolsa' \\
				& kass (kas\textquotesingle s) `gato' 
	\end{tabular}

	Este ablandamiento o palatalización no está indicado en el lenguaje escrito, pero se observa en los diccionarios y listas de palabras como una apóstrofe después de la consonante palatalizada, en los casos en que el significado de la palabra puede ser afectada: pal\textquotesingle k (con suavizado \bemph{i}) `tronco, viga', en comparación con palk (con una \bemph{i} normal no ablandada) `salario'.

\end{enumerate}	

\vfill