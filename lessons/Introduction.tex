% Introduction

\pdfbookmark[1]{Introducción}{Introducción} % Bookmark name visible in a PDF viewer

\chapter*{Introducción} % Introduction name

\begin{enumerate}
	\item El alfabeto estonio básico consta de 23 letras en el siguiente orden:
	\begin{center}
	\begin{otherlanguage}{estonian}
		\textbf{a b c d e g h i j k l m n o p r s t u v õ ä ö ü}
	\end{otherlanguage}
	\end{center}

	Hay otras 9 letras que aparecen en las palabras extranjeras. Las letras \textbf{c q w x y} se encuentran sólo en los nombres extranjeros, como César, Don Quijote, Xantippe, Nueva York. Las letras \textbf{f š z ž} se encuentran en las nuevas palabras prestadas de otros idiomas: film, šokolaad, zooloog, žurnaal. El orden del alfabeto completo es:
	\begin{center}
	\begin{otherlanguage}{estonian}
		\textbf{a b c d e f g h i j k l m n o p q r s š z ž t u v w õ ä ö ü x y}
	\end{otherlanguage}
	\end{center}

	\item En cuanto a la pronunciación, tenga en cuenta lo siguiente:

	\section*{\Large{Vocales}}

	Las vocales \textbf{a e i o u} se pronuncian exactamente igual que en español, sin embargo la fonética de las vocales \textbf{ä õ ö ü} puede presentar una gran dificultad. Es por eso que se recomienda escuchar la pronunciación de las vocales más que leer una descripción. El siguiente \href{http://www.youtube.com/watch?v=GfxrR45yA6I}{video}\footnote{http://www.youtube.com/watch?v=GfxrR45yA6I} presenta de una forma precisa y concisa las vocales del estonio.\\

	Noten que las tres vocales \textbf{õ ö ü} son bastante similares. Una forma de comparar la pronunciación de estas tres vocales es considerar a \textbf{õ} como la pronunciación de una \textbf{o} y una \textbf{u} al mismo tiempo. Para \textbf{ö} es exactamente lo mismo, pero predomina más la \textbf{o}. De manera análoga, para \textbf{ü} predomina más la \textbf{u}.\\

	Si bien ésta es una descripción bastante burda de cómo pronunciar las tres vocales anteriores, da una idea general desde la cual partir.

 	\section*{\Large{Consonantes}}

 	\begin{tabular}{ r p{9.65cm} }

 	\textbf{z} & se pronuncia igual que en inglés.\\[0.5cm]

 	\textbf{j} & se pronuncia como la letra \textbf{y} en la palabra `Maya'.\\[0.5cm] 

 	\textbf{h} & si está al principio de una palabra el sonido es muy débil, casi un silencio. Si está en medio de dos vocales se pronuncia como la \textbf{j} en la interjección coloquial `¡Ajá!', y si se encuentra antes de una consonante o al final de una palabra su pronunciación se torna más como una exhalación muy suave, haciendo amagos en formar una \textbf{j}.\\[0.5cm]

 	\textbf{b d g} & son sordas y ligeramente más suaves que las letras \textbf{p t k} del español, respectivamente.\\[0.5cm]

 	%-----------------------------------
 	% Meramente estético
 	
 	\end{tabular}
 	\newpage 
 	\begin{tabular}{ r p{9.65cm} }
 	%-----------------------------------

 	\textbf{p t k} & son sordas y ligeramente más fuertes que en español. Un sonido más duradero y ligeramente más potente se da en el caso de tener consonantes dobles \textbf{pp tt kk}\\[0.5cm]  

 	\textbf{š} & equivale a la combinación \textbf{sh}, como en la palabra inglesa `english'.\\[0.5cm]

 	\textbf{ž} & es como la \textbf{s} en la palabra inglesa `treasure' o como la \textbf{j} de la palabra francesa `jour'.

 	\end{tabular}

 	Las letras \textbf{l r s f m n v c q w x y} se pronuncian exactamente igual que en español.\\

 	\item En el estonio hay muchos diptongos o combinaciones de dos vocales que forman parte de la misma sílaba. Estos incluyen \textbf{ae ai ao au ea ei eo iu oa oe oi õe äe} y así sucesivamente. Cada una de las dos vocales se pronuncia con claridad, pero no como si estuvieran en sílabas distintas.\\

 	Ejemplos: laud, laev, hea, loen, õun, õed, käed.\\

 	\item La acentuación en el estonio está normalmente en la primera sílaba. Hay, sin embargo, algunas excepciones como \foreignlanguage{estonian}{ai\textbf{täh} `gracias', sõb\textbf{ran}na `amiga', üle\textbf{üld}se `sobre todo'}. En muchas palabras prestadas de otros idiomas, el acento original se ha mantenido también: e\textbf{lek}ter, ide\textbf{aal}, pro\textbf{fes}sor. Cuando la marca \textasciiacute es usada en el texto sobre una vocal (como en elékter, ideáal, proféssor), indica que el acento está sobre esa sílaba, pero esta marca no es parte de la ortografía normal.\\

 	A diferencia del inglés, la sílaba acentuada no domina la pronunciación de la palabra tan notablemente, por lo que las sílabas átonas en palabras estonias son más claras y fáciles de oír que en el Inglés.\\

 	\item La ortografía del estonio es fundamentalmente fonética, lo que significa que las palabras están escritas como suenan. Como una regla básica, letras solas significan sonidos cortos y letras dobles indican sonidos largos.\\

 	La pronunciación de las vocales individuales siempre es muy corto (primer grado), en contraste con las interminables vocales de palabras en inglés como `go', `at', `find'.\\

 	\item Vocales dobles, consonantes dobles y diptongos son largos (segundo grado) o muy largos (tercer grado).\\

 	Cada vocal y consonante en el estonio puede tener por lo tanto tres diferentes largos o grados:\\

 	\begin{tabular}{ r l l c l l}
 		1\textordmasculine\ grado: & s\textbf{a}da 					& `cien' 		& & li\textbf{n}a 					& `mantel, sabana' \\
 		2\textordmasculine\ grado: & s\textbf{aa}da 				& `¡envía!' 	& & li\textbf{nn}a 					& `ciudades' \\
 		3\textordmasculine\ grado: & \textasciigrave s\textbf{aa}da & `obtener' 	& & \textasciigrave li\textbf{nn}a 	& `hacia la ciudad' 
 	\end{tabular}

 	El tercer grado es notablemente más largo que los sonidos correspondientes al inglés.\\

	En la transcripción fonética, el tercero grado se indican con una \textasciigrave antes de la sílaba. Esto no se utiliza en el lenguaje escrito, pero se utiliza en los diccionarios y listas de palabras en los casos en que la longitud afecta el significado. Por ejemplo: \textasciigrave Kooli (pronunciado como si hubieran 3 vocales - koooli), significado de `a la escuela', en comparación con Kooli (pronunciado con sólo dos vocales), que significa `de la escuela'.\\

	\item En algunas palabras, las consonantes \textbf{l n s t} se suavizan o palatalizan con un ligero \textbf{i} o \textbf{j} (la \textbf{j} estonia) antes de la consonante.\\

	\begin{tabular}{ l l}
	Ejemplos: 	& palk (pal\textquotesingle k) `tronco, viga' \\
				& tund (tun\textquotesingle d) `hora' \\
				& kott (kot\textquotesingle t) `bolsa' \\
				& kass (kas\textquotesingle s) `gato' 
	\end{tabular}

	Este ablandamiento o palatalización no está indicado en el lenguaje escrito, pero se observa en los diccionarios y listas de palabras como una apóstrofe después de la consonante palatalizada, en los casos en que el significado de la palabra puede ser afectada: pal\textquotesingle k (con suavizado \textbf{i}) `tronco, viga', en comparación con palk (con una \textbf{i} normal no ablandada) `salario'.

\end{enumerate}	

\vfill